\documentclass{scrartcl}% siehe <http://www.komascript.de>
% packages ---------------------------------------------------------------------------------------------------- packages
\usepackage[margin=2.5cm]{geometry}
\usepackage[utf8]{inputenc}
\usepackage[T1]{fontenc}
\usepackage[ngerman]{babel}
\usepackage{lmodern}
\usepackage{amsmath}
\usepackage{amssymb}
\usepackage{amsfonts}
\usepackage{cancel}
\usepackage{listings}
\usepackage{multirow}
\usepackage{hhline}
\usepackage{tikz}
\usepackage{pgfplots}
\usepackage[all,dvips,arc,curve,color,frame]{xy}
\usepackage[yyyymmdd,hhmm]{datetime}
\usepackage{float}
\usepackage{hyperref}
\usepackage[table]{xcolor,colortbl}
\usepackage{graphicx}
\usepackage{subcaption}
\usepackage{tabularx}
\usepackage{footnote}
\usepackage{multicol}
\usepackage{stackengine}
\usepackage[norndcorners,customcolors,shade]{hf-tikz}
\usepackage{siunitx}
\usepackage{algorithm}
\usepackage[]{algorithmic}
% \usepackage[noend]{algorithmic}

% pgf plotsets -------------------------------------------------------------------------------------------- pgf plotsets
\pgfplotsset{ticks=none}
\pgfplotsset{width=10cm, compat=1.5.1}

% tikz libraries ---------------------------------------------------------------------------------------- tikz libraries
\usetikzlibrary{shapes}
\usetikzlibrary{positioning}
\usetikzlibrary{shapes.misc}
\usetikzlibrary{shapes.arrows,calc,quotes,babel}
\usetikzlibrary{positioning,arrows,chains,scopes,fit}
\usetikzlibrary{matrix,backgrounds}
\usetikzlibrary{trees}
\usetikzlibrary{calc}
\usetikzlibrary{tikzmark}

\sisetup{
locale = DE ,
per-mode = symbol
}

\algsetup{indent=2em}
\newlength\myindent
\setlength\myindent{2em}
\newcommand\bindent{%
    \begingroup
    \setlength{\itemindent}{\myindent}
    \addtolength{\algorithmicindent}{\myindent}
}
\newcommand\eindent{\endgroup}

% commands ---------------------------------------------------------------------------------------------------- commands
\setlength{\parindent}{0pt}
\newcommand{\includepic}[2]{\includegraphics[width=#2\textwidth,height=#2\textheight,keepaspectratio]{#1}}
\newcommand\circlearound[1]{%
\tikz[baseline]\node[draw,shape=circle,anchor=base] {#1} ;}
\newcommand{\emptyrow}[0]{\multicolumn{1}{c}{} & \multicolumn{1}{c}{} & \multicolumn{1}{c}{} & \multicolumn{1}{c}{} &
\multicolumn{1}{c}{} & \multicolumn{1}{c}{} & \multicolumn{1}{c}{} & \multicolumn{1}{c}{} &
\multicolumn{1}{c}{} & \multicolumn{1}{c|}{} & & \\}
\newcommand\underbracetext[1]{\raisebox{1ex}{\ensuremath{\underbrace{\hphantom{\text{#1}}}_{\text{\normalsize #1}}}}}
\newcommand{\headerline}[3]{
\subject{#2}
\title{#1}
\subtitle{#3}
\date{Letztes Update: \today \ - \currenttime \ Uhr}
\maketitle
}
\newcommand{\newproof}[2]{
\vspace*{0.3cm} \textbf{\textsf{Satz:}} #1

\vspace*{0.3cm} \textbf{\textsf{Beweis:}} #2 \vspace*{0.3cm}
}
\newcommand{\uparrow}[0]{\rlap{\scalebox{1.6}{$\uparrow$}}}
\newlength\dlf
\newcommand{\proofend}[0]{\hfill $\square$}
\newcommand\drawbi[7][4pt]{%
    \path(#2)--(#3)coordinate[at start](h1)coordinate[at end](h2);
    \draw[#4]($(h1)!#1!90:(h2)$)-- node [auto=left] {#5} ($(h2)!#1!-90:(h1)$);
    \draw[#6]($(h1)!#1!-90:(h2)$)-- node [auto=right] {#7} ($(h2)!#1!90:(h1)$);
    }

% colors -------------------------------------------------------------------------------------------------------- colors
\definecolor{amethyst}{rgb}{0.6, 0.4, 0.8}
\definecolor{anti-flashwhite}{rgb}{0.95, 0.95, 0.96}
\definecolor{applegreen}{rgb}{0.55, 0.71, 0.0}
\definecolor{babyblue}{rgb}{0.54, 0.81, 0.94}
\definecolor{bistre}{rgb}{0.24, 0.17, 0.12}
\definecolor{buff}{rgb}{0.94, 0.86, 0.51}
\definecolor{dodgerblue}{rgb}{0.12, 0.56, 1.0}
\definecolor{icterine}{rgb}{0.99, 0.97, 0.37}
\definecolor{tan}{rgb}{0.82, 0.71, 0.55}
\definecolor{upforestgreen}{rgb}{0.0, 0.27, 0.13}

% listing colors
\definecolor{listingBlue}{rgb}{0.13,0.13,1}
\definecolor{listingGreen}{rgb}{0,0.5,0}
\definecolor{listingRed}{rgb}{0.9,0,0}
\definecolor{listingGrey}{rgb}{0.46,0.45,0.48}

% listings ---------------------------------------------------------------------------------------------------- listings
\lstset{basicstyle=\ttfamily}
\lstset{literate=%
{Ö}{{\"O}}1
{Ä}{{\"A}}1
{Ü}{{\"U}}1
{ß}{{\ss}}1
{ü}{{\"u}}1
{ä}{{\"a}}1
{ö}{{\"o}}1
}
\lstset{language=Java,
showspaces=false,
showtabs=false,
breaklines=true,
showstringspaces=false,
breakatwhitespace=true,
commentstyle=\color{listingGreen},
keywordstyle=\color{listingBlue},
stringstyle=\color{listingRed},
basicstyle=\ttfamily,
moredelim=[il][\textcolor{listingGrey}]{$$},
moredelim=[is][\textcolor{listingGrey}]{\%\%}{\%\%}
}
\lstdefinelanguage{Kotlin}{
  keywords={package, as, typealias, this, super, val, var, fun, for, null, true, false, is, in, throw, return, break, continue, object, if, try, else, while, do, when, yield, typeof, yield, typeof, class, interface, enum, object, override, public, private, get, set, import, abstract, },
  keywordstyle=\color{listingBlue}\bfseries,
  ndkeywords={@Deprecated, Iterable, Int, Integer, Float, Double, String, Runnable, dynamic},
  ndkeywordstyle=\color{amethyst}\bfseries,
  emph={println, return@, forEach,},
  emphstyle={\color{orange}},
  identifierstyle=\color{black},
  sensitive=true,
  commentstyle=\color{listingGreen}\ttfamily,
  comment=[l]{//},
  morecomment=[s]{/*}{*/},
  stringstyle=\color{listingRed}\ttfamily,
  morestring=[b]",
  morestring=[s]{"""*}{*"""},
  numbers=left,
}

% misc ------------------------------------------------------------------------------------------------------------ misc
\graphicspath{ {../lectures-img/} }
\setlength{\parindent}{0pt}

\newcolumntype{C}[1]{>{\centering\arraybackslash}p{#1}} % zentriert mit Breitenangabe
\newcolumntype{L}[1]{>{\raggedright\arraybackslash}p{#1}} % rechtsbündig mit Breitenangabe
\newcolumntype{R}[1]{>{\raggedleft\arraybackslash}p{#1}} % rechtsbündig mit Breitenangabe


\usepackage{diagbox}

\definecolor{cadmiumred}{rgb}{0.89, 0.0, 0.13}
\definecolor{cadmiumorange}{rgb}{0.93, 0.53, 0.18}
\definecolor{cadmiumyellow}{rgb}{1.0, 0.96, 0.0}
\definecolor{cadmiumgreen}{rgb}{0.0, 0.42, 0.24}
\definecolor{blizzardblue}{rgb}{0.67, 0.9, 0.93}
\definecolor{persianblue}{rgb}{0.11, 0.22, 0.73}
\definecolor{darkviolet}{rgb}{0.58, 0.0, 0.83}

\begin{document}
    \headerline{Algorithmen und Berechenbarkeit}{Vorlesungsmitschrift}{Vorlesung 16}

    \section*{Das Allgemeine Halteproblem: Die Sprache $\mathcal{H}_{all}$}
    Die Sprache $\mathcal{H}_{all}$ ist definiert als

    \begin{equation*}
        \mathcal{H}_{all} = \{ \text{<M>}\ |\ \text{ M hält auf allen Eingaben} \}
    \end{equation*}

    Das Allgemeine Halteproblem besteht aus den beiden Teilsätzen,
    dass sowohl $\mathcal{H}_{all}$ als auch $\overline{\mathcal{H}_{all}}$ nicht semi-entscheidbar sind.

    \vspace*{1cm}
    \textbf{\textsf{Satz:}} $\overline{\mathcal{H}_{all}}$ ist nicht semi-entscheidbar.

    \vspace*{0.3cm}
    \textbf{\textsf{Beweis:}} Für den Beweis wird eine Sprache $\mathcal{X}$ gesucht, die nicht semi-entscheidbar ist.
    Anschließend wird  $\mathcal{X} \leq \overline{\mathcal{H}_{all}}$.

    Für $\mathcal{X}$ eignen sich einige Sprachen: Es wurde bereits gezeigt, dass $\mathcal{H}_\epsilon$ semi-entscheidbar aber nicht entscheidbar ist ($\mathcal{H}_\epsilon$ ist unentscheidbar und semi-entscheidbar).
    Deshalb ist auch bekannt, dass $\overline{\mathcal{H}_\epsilon}$ nicht semi-entscheidbar ist.

    \vspace*{0.5cm}
    \par
    \begingroup
    \leftskip=1cm % Parameter anpassen
    \noindent

    \vspace*{0.3cm}
    \textbf{\textsf{Einschub: Entscheidbarkeit/Unentscheidbarkeit - Semi-entscheidbar}}

    \par
    \endgroup
    \par
    \begingroup
    \leftskip=1cm % Parameter anpassen
    \noindent

    \begin{figure}[H]
        \centering
        \begin{tikzpicture}[
        thick]
            \draw [fill=cadmiumgreen, fill opacity=0.7, name path=c1] (0,0) circle (2cm);
            \draw [fill=cadmiumred, fill opacity=0.3, name path=c2] (3,0) circle (2.5cm);
            \draw (0,0) ++(120:2cm) -- ++(150:2.2cm) node [fill=white,inner sep=5pt](a){Entscheidbare Sprachen};
            \draw (3, 0) ++(30:2.5cm) -- ++(45 :2.6cm) node [fill=white,inner sep=5pt](b){Unentscheidbare-Sprachen};
            \path [name intersections={of=c1 and c2,by=cs}];
            \draw (cs) -- ++(0,2) node [fill=white,inner sep=5pt](c){Un- aber semi-entscheidbare Sprachen};
        \end{tikzpicture}
    \end{figure}
    \par
    \endgroup
    \par
    \begingroup
    \leftskip=1cm % Parameter anpassen
    \noindent
    \newpage
    Daraus ergeben sich beispielhaft die folgenden Zusammenhänge
    \begin{itemize}
        \item [$\rightarrow$] Wenn entscheidbar $\Rightarrow$ auch semi-entscheidbar
        \item [$\rightarrow$] Wenn semi-entscheidbar $\centernot\Rightarrow$ entscheidbar
        \item [$\rightarrow$] Wenn unentscheidbar $\centernot\Rightarrow$ semi-entscheidbar
        \item [$\rightarrow$] Wenn nicht semi-entscheidbar $\Rightarrow$ unentscheidbar
    \end{itemize}
    \par
    \endgroup

    \vspace*{0.3cm}
    \textbf{\textsf{Beweis-Fortführung:}} Wählt man nun $\mathcal{X} = \overline{\mathcal{H}_\epsilon}$
    (man zeigt also $\overline{\mathcal{H}_\epsilon} \leq \overline{\mathcal{X}_{all}}$) wird eine Funktion
    $f'$ benötigt, sodass gilt

    \begin{equation*}
        x \in \overline{\mathcal{H}_\epsilon} \quad \Leftrightarrow \quad f(x) \in \overline{\mathcal{H}_{all}}
    \end{equation*}

    oder äquivalent

    \begin{equation*}
        x \in \mathcal{H}_\epsilon \quad \Leftrightarrow \quad f(x) \in \mathcal{H}_{all}
    \end{equation*}

    und damit

    \begin{equation*}
        \mathcal{H}_\epsilon \leq \mathcal{H}_{all}
    \end{equation*}

    Die Funktion $f$ wird wie folgt definiert bzw. berechnet: Sei $w$ die Eingabe für $\mathcal{H}_\epsilon$.
    \begin{itemize}
        \item Falls $w$ keine gültige TM-Kodierung, dann setzt man $f(w) = w$.
        \item Falls $w$ eine gültige TM-Kodierung <M>, dann setzt man $f(w) = \text{<M}^{*}_\epsilon\text{>}$ mit der folgenden Eigenschaft:
        $\text{<M}^{*}_\epsilon\text{>}$ ignoriert und simuliert $\mathcal{M}$ mit der Eingabe $\epsilon$.
    \end{itemize}

    Korrektheit der Konstruktion:
    \begin{itemize}
        \item [] Falls $w$ keine TM-Kodierung $\Rightarrow w \notin \mathcal{H}_\epsilon$ und $f(w) \notin \mathcal{H}_{all}$.\newline
        Andernfalls gilt $w=$<M> und $f(w)=\text{<M}^{*}_\epsilon\text{>}$.
        \item $w \in \mathcal{H}_\epsilon$
        \begin{itemize}
            \item [$\Rightarrow$] $\mathcal{M}$ hält auf der Eingabe $\epsilon$
            \item [$\Rightarrow$] $\text{<M}^{*}_\epsilon\text{>}$ hält auf allen Eingaben
            \item [$\Rightarrow$] $f(w) \in \mathcal{H}_{all}$
        \end{itemize}
        \item $w \notin \mathcal{H}_\epsilon$
        \begin{itemize}
            \item [$\Rightarrow$] $\mathcal{M}$ hält nicht auf der Eingabe $\epsilon$
            \item [$\Rightarrow$] $\text{<M}^{*}_\epsilon\text{>}$ hält nie
            \item [$\Rightarrow$] $f(w) \notin \mathcal{H}_{all}$
        \end{itemize}
    \end{itemize}

    Somit gilt:
    \begin{align*}
        w \in \mathcal{H}_{\epsilon} & \quad \Leftrightarrow \quad  f(w) \in \mathcal{H}_{all}, \text{ also ist} \\
        \mathcal{H}_{\epsilon} \leq \mathcal{H}_{all} & \quad \Rightarrow \quad  \overline{\mathcal{H}_{\epsilon}} \leq \overline{\mathcal{H}_{all}}\\
        & \quad \Rightarrow \quad  \overline{\mathcal{H}_{all}} \text{ nicht entscheidbar}\\
    \end{align*}\proofend

    \newpage
    \textbf{\textsf{Satz:}} $\mathcal{H}_{all}$ ist nicht semi-entscheidbar.

    \vspace*{0.3cm}
    \textbf{\textsf{Beweis:}}
    Es wird wieder mit $\overline{\mathcal{H}_\epsilon} \leq \mathcal{H}_{all}$ argumentiert.
    Man konstruiert eine Funktion $f$, die \texttt{JA}-Instanzen von $\overline{\mathcal{H}_\epsilon}$ auf \texttt{JA}-Instanzen von
    $\mathcal{H}_{all}$ und \texttt{NEIN}-Instanzen von $\overline{\mathcal{H}_\epsilon}$ auf \texttt{NEIN}-Instanzen von
    $\mathcal{H}_{all}$ abbildet.

    Sei $w$ die Eingabe für $\overline{\mathcal{H}_\epsilon}$
    \begin{itemize}
        \item Falls $w$ keine gültige TM-Kodierung, dann gilt $w \in \overline{\mathcal{H}_\epsilon}$. \newline Außerdem wird $f(w)= \text{ Kodierung einer TM in } \mathcal{H}_{all}$.
        \item Falls $w$ eine gültige TM-Kodierung <M>, dann berechnet man eine Kodierung $\text{<M}^{'}_{\mathcal{M}}\text{>}$ einer TM $\mathcal{M}'_{\mathcal{M}}$,
        die auf der Eingabe $x$ Folgendes tut:
        Falls $|x|=i$, simuliert sie die ersten $i$-Schritte von $\mathcal{M}$ auf der Eingabe $\epsilon$.
        Wenn $\mathcal{M}$ dabei hält, geht sie in eine Endlosschleife, ansonsten hält sie an: $f(w)=\text{<M}^{'}_{\mathcal{M}}\text{>}$
    \end{itemize}

    Korrektheit der Konstruktion:
    \begin{itemize}
        \item $w \in \overline{\mathcal{H}_\epsilon}$
        \begin{itemize}
            \item [$\Rightarrow$] $\mathcal{M}$ hält nicht auf der Eingabe $\epsilon$
            \item [$\Rightarrow$] $\neg \exists i: \mathcal{M}$ hält innerhalb der ersten $i$-Schritte auf $\epsilon$
            \item [$\Rightarrow$] $\forall i: \text{<M}^{'}_{\mathcal{M}}\text{>}$ hält auf der Eingabe der Länge $i$
            \item [$\Rightarrow$] $f(w) = \text{<M}^{'}_{\mathcal{M}}\text{>} \in \mathcal{H}_{all}$
        \end{itemize}
        \item $w \notin \overline{\mathcal{H}_\epsilon}$
        \begin{itemize}
            \item [$\Rightarrow$] $\mathcal{M}$ hält auf der Eingabe $\epsilon$
            \item [$\Rightarrow$] $\exists i: \mathcal{M}$ hält innerhalb der ersten $i$-Schritte auf $\epsilon$
            \item [$\Rightarrow$] $\exists i: \text{<M}^{'}_{\mathcal{M}}\text{>}$ loopt auf Eingaben der Länge $i$
            \item [$\Rightarrow$] $f(w) = \text{<M}^{'}_{\mathcal{M}}\text{>} \notin \mathcal{H}_{all}$
        \end{itemize}
    \end{itemize}

    Somit gilt:
    \begin{align*}
        w \in \overline{\mathcal{H}_{\epsilon}} & \quad \Leftrightarrow \quad f(w) \in \mathcal{H}_{all}
    \end{align*}\proofend

    \vspace*{0.3cm}
    Es wurde also gezeigt, dass sowohl $\mathcal{H}_{all}$ als auch $\overline{\mathcal{H}_{all}}$ nicht semi-entscheidbar bzw. rekursiv aufzählbar sind.

    \newpage
    \section*{Weitere unentscheidbare Probleme}
    \subsection*{Hilberts zehntes Problem}
    Gegeben sei ein multivariates Polynom $p$ (also ein Polynom mit mehreren Variablen z.B. $x,y,z,a \dots$).
    Es kann die Frage gestellt werden, ob $p$ eine ganzzahlige Nullstelle besitzt.

    \vspace*{0.3cm}
    In der Tat gilt: Die Sprache

    \begin{align*}
        N = \{\ p\ |\ & p \text{ ist Polynom mit ganzzahligen Koeffizienten und ganzzahliger Nullstelle } \}
    \end{align*}
    ist unentscheidbar.

    \subsection*{Postsche Korrespondenzproblem}
    Gegeben seien z.B. die Kärtchen

    \begin{equation*}
        K = \left\{
        \underbrace{\left[ \frac{b}{ca}\right]}_{1},
        \underbrace{\left[ \frac{a}{ab}\right]}_{2},
        \underbrace{\left[ \frac{ca}{a}\right]}_{3},
        \underbrace{\left[ \frac{abc}{c}\right]}_{4}
        \right\}
    \end{equation*}

    Man kann nun beliebig viele Kärtchen beliebig oft aneinanderreihen.
    Reiht man die Kärtchen $1+2+1$ aneinander, so erhält man zwei Zeichenketten: Einmal die Zeichenkette oben und einmal die Zeichenkette unten.
    Im Beispiel also oben: \texttt{bab} und unten: \texttt{caabca}.

    \vspace*{0.3cm}
    Nun kann die Frage gestellt werden, ob man bei einem gegebenen Kartenset $K$ die Karten so anordnen kann,
    dass die obere Zeichenkette genau der unteren Zeichenkette entspricht.
    Für das obige Beispiel ist das möglich, dort liefert die Kombination $2+1+3+2+4$ sowohl oben als auch unten dieselbe Zeichenkette.

    \vspace*{0.3cm}
    Im Allgemeinen gilt aber: Die Sprache

    \begin{equation*}
        PKP = \{ \text{ Kärtchenset } K \text{ mit Lösung } \}
    \end{equation*}

    ist unentscheidbar (aber semi-entscheidbar).

    \vspace*{0.3cm}
    \textit{Im Skript weiter informieren, eine Klausuraufgabe hierzu ist sehr gut möglich.}

\end{document}
