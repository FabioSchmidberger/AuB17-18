\documentclass{scrartcl}% siehe <http://www.komascript.de>
\usepackage[margin=2.5cm]{geometry}
\usepackage[utf8]{inputenc}
\usepackage[T1]{fontenc}
%\usepackage[ngerman]{babel}
\usepackage{lmodern}
\usepackage{amsmath}
\usepackage{amssymb}
\usepackage{cancel}
%\usepackage{mathrsfs}
%\usepackage{mathtools}
\usepackage{listings}
\usepackage{tikz}
\usepackage{pgfplots}
\usetikzlibrary{shapes.misc}
\usetikzlibrary{matrix,backgrounds}
\usetikzlibrary{trees}
\usetikzlibrary{calc}
\usepackage[all,dvips,arc,curve,color,frame]{xy}
\usepackage[yyyymmdd,hhmm]{datetime}


\lstset{basicstyle=\ttfamily}
\lstset{literate=%
{Ö}{{\"O}}1
{Ä}{{\"A}}1
{Ü}{{\"U}}1
{ß}{{\ss}}1
{ü}{{\"u}}1
{ä}{{\"a}}1
{ö}{{\"o}}1
}

\newtheorem{theorem}{Theorem}

\newcommand{\includepic}[2]{\includegraphics[width=#2\textwidth,height=#2\textheight,keepaspectratio]{#1}}

\usepackage[yyyymmdd,hhmm]{datetime}

% Base path of images
\usepackage{graphicx}
\graphicspath{ {../lectures-img/} }

\begin{document}
    % ----------------------------------------------------------------------------
    \subject{Vorlesungsmitschrieb}
    \title{Algorithmen und Berechenbarkeit}
    \subtitle{Vorlesung 05}
    \date{Letztes Update: \today \ - \currenttime \ Uhr}
    \maketitle
    % ----------------------------------------------------------------------------

    \section*{Quick-Select (Randomisiert)}\label{sec:quick-select}
    Es soll in einem unsortierten Array bzw. einer unsortierten Liste $A$ das $k$-kleinste Element gefunden werden.
    Sei $A$ zum Beispiel [$3,5,1,4,7$] und $k=2$, so soll der Algorithmus den Wert $3$ zurückgeben, was in der Eingabe dem $k$-kleinsten Wert entspricht.

    Spontan und einfach zu implementieren ist vor allem diese Methode: Sortiere die Eingabedatenstruktur mit Quick- bzw.
    Merge-Sort und gebe $A[i]$ zurück.
    Bei diesem Ansatz in die Laufzeit maßgeblich vom verwendeten Sortieralgorithmus abhängig.
    Es fällt außerdem auf, dass eine sortierte Liste die Antwort nun für jedes $k$ sehr schnell liefern kann.
    Das bedeutet, der Algorithmus ist nicht auf das Nötigste beschränkt und \textit{macht zu viel}.

    Eine einfache Optimierung erschließt sich beim Betrachten der Skizze: Ist bereits bekannt, dass sich das $k$-te Element nicht im betrachteten Abschnitt der Datenstruktur aufhält, so muss dieser Teil nicht sortiert werden (gelb).
    \begin{figure}[htb]
        \centering
        \includepic{lec_05_a}{0.4}
    \end{figure}

    \begin{lstlisting}[escapeinside={(;}{;)}]
        Select((;$A, k$;)):
            Wähle Pivotelement p zufällig gleichverteilt
            (;$A_1$;) := Array-Teil bis Index p
            (;$A_2$;) := Array-Teil ab Index p

            if (;$k \leq A_1$;).length
                Select((;$A_1, k$;))
            elif (;$k > A_2$;).length
                Select((;$A_2, k-A_1$;).length)
    \end{lstlisting}

    \subsection*{Laufzeit von Quick-Select (Randomisiert)}\label{subsec:laufzeitVonQuick-selectrandomisiert}
    \begin{itemize}
        \item Fall immer der Median getroffen wird, so ergibt sich $n+\frac{1}{2}+\frac{1}{4}\ ... \leq 2n$
        \item Mit $X_{ij}$ wie bei Quicksort: $E[X_{ij}] = \frac{2}{j-i+1}$.
        \item [] Es gibt nun die drei Fälle
        \begin{itemize}
            \item $k<i<j$
            \item $i<k<j$
            \item $i<j<k$
        \end{itemize}
        \item [\Rightarrow] Die Laufzeit ist $\mathcal{O}(n)$.
        Schneller ist nicht möglich, da jedes Element betrachtet werden muss.
    \end{itemize}

    \section*{Quick-Select (Deterministisch)}\label{sec:quick-selectdeterministisch}
    In einigen Anwendungsszenarien ist es wichtiger, dass ein Ergebnis konstant schnell vorliegt als \textit{wahrscheinlich schnell}.

    Wurde im randomisierten Algorithmus für Quick-Select die Wahl des Pivotelements dem Zufall überlassen,
    so soll das Pivotelement nun nach einem festen Ablauf ermittelt werden.
    Dabei ist es wichtig, dass das Pivotelement nicht perfekt gewählt sein muss, sondern nur \textit{ausreichend gut}.

    \vspace{0.3cm}
    \textsf{\textbf{Wahl des Pivotelements}}:
    \begin{enumerate}
        \item Man gruppiere die Elemente der Eingabestruktur zu je 5 Elementen.
        \item Man berechnet in jeder Gruppe den Median (geht in sechs Vergleichen: $\mathcal{O}(n)$).
        Nun hat man $\frac{n}{5}$ Gruppenmediane ($M[i] := \text{Median der Gruppe }i$).
        \item Man berechnet \textbf{rekursiv} den Median der Gruppenmediane: \texttt{Select($M, \dfrac{\frac{n}{5}}{2}$)}
        \item Der Median der Gruppenmediane wird als Pivotelement verwendet
    \end{enumerate}

    \subsubsection*{Pivotelement gut genug}
    \begin{itemize}
        \item Sei $M$ der Median der Gruppenmediane.
        Von den $\frac{n}{5}$ Gruppenmedianen sind die Hälfte $(\frac{n}{5}:2=\frac{n}{10})$ größer bzw. kleiner als $M$.
        \item In jeder Gruppe, deren Gruppenmedian $<M$ ist, sind 3 Elemente $<M$ bzw. in jeder Gruppe, deren Gruppenmedian $>M$ ist, sind 3 Elemente $>M$.
        \item Zusammen ergibt das
        \begin{itemize}
            \item [$\rightarrow$] $3 \cdot \frac{n}{10} \text{ Elemente }<M$ oder eben
            \item [$\rightarrow$] $3 \cdot \frac{n}{10} \text{ Elemente }>M$
        \end{itemize}
        \item [$\Rightarrow$] Wählt man also nach dieser Methode das Pivotelement, so erhält man mindestens einen $30\%-70\%$ Split.
    \end{itemize}

    \subsubsection*{Pivotelement schnell genug gefunden}
    Für die Laufzeit ergibt sich
    \begin{equation*}
        T(n) \leq \underbrace{T(\dfrac{n}{5})}_{\substack{\text{Median} \\ \text{der} \\ \text{Mediane}}}
        + \underbrace{T(\dfrac{7}{10} \cdot n)}_{\substack{\text{Quick-Sort-} \\ \text{Rekursion}}}
        + \underbrace{\mathcal{O}(n)}_{\substack{\text{Pivotieren} \\ \text{+} \\ \text{Verwaltung}}}
    \end{equation*}
    Per Induktion kann nun gezeigt werden $T(n)\leq \gamma \cdot n$.
    Im Induktionsschritt zeigt sich dann
    \begin{equation*}
        \begin{flalign}
            T(n) & \overset{\text{IV}}{\leq} \gamma \cdot \frac{n}{5} + \gamma \cdot \frac{7}{10} \cdot n + c \cdot n\\\nonumber
            &= n \cdot (\frac{9}{10} \cdot \gamma + c)\\\nonumber
            &\leq \gamma \cdot n
        \end{flalign}
    \end{equation*}
    Dies ist der Fall für $\gamma = 10 \cdot c\qquad \Longrightarrow T(n) \in \mathcal{O}(n)$

    \textbf{\textsf{Nebeneffekt:}} Der Median wird in Linearzeit berechnet (im Beispiel des Quick-Sort-Pivots).

    \section*{Concentration Bounds}\label{sec:concentrationBounds}
    Bisher wurden hauptsächlich Analysen mithilfe des Erwartungswerts durchgeführt.
    Der Erwartungswert macht jedoch keine Aussage über die Verteilung.

    \begin{figure}[htb]
        \centering
        \includepic{lec_05_b}{0.5}
    \end{figure}

    Man kann die Verteilung eines Algorithmus mit Erwartungswert $L$ generisch analysieren: Der Algorithmus läuft $\alpha \cdot L$ Schritte.
    Ist er danach nicht fertig, wird er einfach neugestartet.
    Die Wahrscheinlichkeit, in einem Durchlauf nicht fertig zu werden, ist $<\frac{1}{\alpha}$.
    Die Wahrscheinlichkeit, in $k$ Durchläufen nicht fertig zu werden, ist $\leq \frac{1}{\alpha} \cdot \frac{1}{\alpha} \cdot\ ... = \frac{1}{\alpha^k}$.

    \vspace{0.3cm}
    \textsf{\textbf{Beispiel}}: Gegeben ist begrenzte Zeit $t \underbrace{>>}_{\substack{\text{viel} \\ \text{größer}}} L$
    \begin{itemize}
        \item Wie soll $\alpha$ gewählt werden?
        \item Für gegebenes $\alpha$ können maximal $\frac{t}{\alpha \cdot l}$ Runden durchlaufen werden.
    \end{itemize}


\end{document}