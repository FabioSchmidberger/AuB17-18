\documentclass{scrartcl}% siehe <http://www.komascript.de>
% packages ---------------------------------------------------------------------------------------------------- packages
\usepackage[margin=2.5cm]{geometry}
\usepackage[utf8]{inputenc}
\usepackage[T1]{fontenc}
\usepackage[ngerman]{babel}
\usepackage{lmodern}
\usepackage{amsmath}
\usepackage{amssymb}
\usepackage{amsfonts}
\usepackage{cancel}
\usepackage{listings}
\usepackage{multirow}
\usepackage{hhline}
\usepackage{tikz}
\usepackage{pgfplots}
\usepackage[all,dvips,arc,curve,color,frame]{xy}
\usepackage[yyyymmdd,hhmm]{datetime}
\usepackage{float}
\usepackage{hyperref}
\usepackage[table]{xcolor,colortbl}
\usepackage{graphicx}
\usepackage{subcaption}
\usepackage{tabularx}
\usepackage{footnote}
\usepackage{multicol}
\usepackage{stackengine}
\usepackage[norndcorners,customcolors,shade]{hf-tikz}
\usepackage{siunitx}
\usepackage{algorithm}
\usepackage[]{algorithmic}
% \usepackage[noend]{algorithmic}

% pgf plotsets -------------------------------------------------------------------------------------------- pgf plotsets
\pgfplotsset{ticks=none}
\pgfplotsset{width=10cm, compat=1.5.1}

% tikz libraries ---------------------------------------------------------------------------------------- tikz libraries
\usetikzlibrary{shapes}
\usetikzlibrary{positioning}
\usetikzlibrary{shapes.misc}
\usetikzlibrary{shapes.arrows,calc,quotes,babel}
\usetikzlibrary{positioning,arrows,chains,scopes,fit}
\usetikzlibrary{matrix,backgrounds}
\usetikzlibrary{trees}
\usetikzlibrary{calc}
\usetikzlibrary{tikzmark}

\sisetup{
locale = DE ,
per-mode = symbol
}

\algsetup{indent=2em}
\newlength\myindent
\setlength\myindent{2em}
\newcommand\bindent{%
    \begingroup
    \setlength{\itemindent}{\myindent}
    \addtolength{\algorithmicindent}{\myindent}
}
\newcommand\eindent{\endgroup}

% commands ---------------------------------------------------------------------------------------------------- commands
\setlength{\parindent}{0pt}
\newcommand{\includepic}[2]{\includegraphics[width=#2\textwidth,height=#2\textheight,keepaspectratio]{#1}}
\newcommand\circlearound[1]{%
\tikz[baseline]\node[draw,shape=circle,anchor=base] {#1} ;}
\newcommand{\emptyrow}[0]{\multicolumn{1}{c}{} & \multicolumn{1}{c}{} & \multicolumn{1}{c}{} & \multicolumn{1}{c}{} &
\multicolumn{1}{c}{} & \multicolumn{1}{c}{} & \multicolumn{1}{c}{} & \multicolumn{1}{c}{} &
\multicolumn{1}{c}{} & \multicolumn{1}{c|}{} & & \\}
\newcommand\underbracetext[1]{\raisebox{1ex}{\ensuremath{\underbrace{\hphantom{\text{#1}}}_{\text{\normalsize #1}}}}}
\newcommand{\headerline}[3]{
\subject{#2}
\title{#1}
\subtitle{#3}
\date{Letztes Update: \today \ - \currenttime \ Uhr}
\maketitle
}
\newcommand{\newproof}[2]{
\vspace*{0.3cm} \textbf{\textsf{Satz:}} #1

\vspace*{0.3cm} \textbf{\textsf{Beweis:}} #2 \vspace*{0.3cm}
}
\newcommand{\uparrow}[0]{\rlap{\scalebox{1.6}{$\uparrow$}}}
\newlength\dlf
\newcommand{\proofend}[0]{\hfill $\square$}
\newcommand\drawbi[7][4pt]{%
    \path(#2)--(#3)coordinate[at start](h1)coordinate[at end](h2);
    \draw[#4]($(h1)!#1!90:(h2)$)-- node [auto=left] {#5} ($(h2)!#1!-90:(h1)$);
    \draw[#6]($(h1)!#1!-90:(h2)$)-- node [auto=right] {#7} ($(h2)!#1!90:(h1)$);
    }

% colors -------------------------------------------------------------------------------------------------------- colors
\definecolor{amethyst}{rgb}{0.6, 0.4, 0.8}
\definecolor{anti-flashwhite}{rgb}{0.95, 0.95, 0.96}
\definecolor{applegreen}{rgb}{0.55, 0.71, 0.0}
\definecolor{babyblue}{rgb}{0.54, 0.81, 0.94}
\definecolor{bistre}{rgb}{0.24, 0.17, 0.12}
\definecolor{buff}{rgb}{0.94, 0.86, 0.51}
\definecolor{dodgerblue}{rgb}{0.12, 0.56, 1.0}
\definecolor{icterine}{rgb}{0.99, 0.97, 0.37}
\definecolor{tan}{rgb}{0.82, 0.71, 0.55}
\definecolor{upforestgreen}{rgb}{0.0, 0.27, 0.13}

% listing colors
\definecolor{listingBlue}{rgb}{0.13,0.13,1}
\definecolor{listingGreen}{rgb}{0,0.5,0}
\definecolor{listingRed}{rgb}{0.9,0,0}
\definecolor{listingGrey}{rgb}{0.46,0.45,0.48}

% listings ---------------------------------------------------------------------------------------------------- listings
\lstset{basicstyle=\ttfamily}
\lstset{literate=%
{Ö}{{\"O}}1
{Ä}{{\"A}}1
{Ü}{{\"U}}1
{ß}{{\ss}}1
{ü}{{\"u}}1
{ä}{{\"a}}1
{ö}{{\"o}}1
}
\lstset{language=Java,
showspaces=false,
showtabs=false,
breaklines=true,
showstringspaces=false,
breakatwhitespace=true,
commentstyle=\color{listingGreen},
keywordstyle=\color{listingBlue},
stringstyle=\color{listingRed},
basicstyle=\ttfamily,
moredelim=[il][\textcolor{listingGrey}]{$$},
moredelim=[is][\textcolor{listingGrey}]{\%\%}{\%\%}
}
\lstdefinelanguage{Kotlin}{
  keywords={package, as, typealias, this, super, val, var, fun, for, null, true, false, is, in, throw, return, break, continue, object, if, try, else, while, do, when, yield, typeof, yield, typeof, class, interface, enum, object, override, public, private, get, set, import, abstract, },
  keywordstyle=\color{listingBlue}\bfseries,
  ndkeywords={@Deprecated, Iterable, Int, Integer, Float, Double, String, Runnable, dynamic},
  ndkeywordstyle=\color{amethyst}\bfseries,
  emph={println, return@, forEach,},
  emphstyle={\color{orange}},
  identifierstyle=\color{black},
  sensitive=true,
  commentstyle=\color{listingGreen}\ttfamily,
  comment=[l]{//},
  morecomment=[s]{/*}{*/},
  stringstyle=\color{listingRed}\ttfamily,
  morestring=[b]",
  morestring=[s]{"""*}{*"""},
  numbers=left,
}

% misc ------------------------------------------------------------------------------------------------------------ misc
\graphicspath{ {../lectures-img/} }
\setlength{\parindent}{0pt}

\newcolumntype{C}[1]{>{\centering\arraybackslash}p{#1}} % zentriert mit Breitenangabe
\newcolumntype{L}[1]{>{\raggedright\arraybackslash}p{#1}} % rechtsbündig mit Breitenangabe
\newcolumntype{R}[1]{>{\raggedleft\arraybackslash}p{#1}} % rechtsbündig mit Breitenangabe


\begin{document}
    \headerline{Algorithmen und Berechenbarkeit}{Vorlesungsmitschrift}{Vorlesung 10}

    \section*{Routenplanung in Straßengraphen}
    Gegeben ist ein Graph $G(V, E, c)$, bei dem wie gehabt $V$ die Menge der Knoten und $E$ die Menge der Kanten darstellt.
    Im Gegensatz zum herkömmlichen Graph kommt hier noch die Reisezeit $c$ hinzu.
    Ein Straßengraph kann wie im Folgenden dargestellt werden

    \begin{figure}[htb]
        \centering
        \begin{tikzpicture}[myn/.style={circle,very thick,draw,inner sep=0.25cm,outer sep=3pt, fill=anti-flashwhite!100}]
            \node[myn] (a) at (0,2) {};
            \node[myn] (b) at (3,0) {};
            \node[myn] (c) at (3,4) {};
            \node[myn] (d) at (6,0) {};
            \node[myn] (e) at (6,4) {};

            \draw[->, very thick] (a) -- (b) node[midway,sloped, below] {\SI{3}{\m}};
            \draw[->, very thick] (b) -- (c) node[midway,sloped, right, rotate=270] {\SI{3}{\m}};

            \drawbi{a}{c}{->,very thick}{\SI{5}{\m}}{<-,very thick}{\SI{3}{\m}}
            \drawbi{b}{d}{->,very thick}{\SI{5}{\m}}{<-,very thick}{\SI{4}{\m}}
            \drawbi{c}{e}{->,very thick}{\SI{1}{\m}}{<-,very thick}{\SI{2}{\m}}
            \drawbi{d}{e}{->,very thick}{\SI{1}{\m}}{<-,very thick}{\SI{2}{\m}}
        \end{tikzpicture}
    \end{figure}

    Eine Anfrage beinhaltet einen Startknoten $s$ und einen Targetknoten $t$ wobei $s,t \in V$.
    Das Ziel ist der kürzeste/schnellste Pfad von $s$ nach $t$.
    Für dieses Problem gibt es verschiedene Algorithmen, das Standardverfahren wäre \textit{Dijkstra}.
    Bei einem Straßengraphen von Deutschland (Größenordnung: $|V| \approx 20 \text{\,Millionen}, |E| \approx 40 \text{\,Millionen}$) dauert eine Anfrage etwa \SI{5}{\s},
    wenn der Dijkstra-Algorithmus ordentlich implementiert wurde.

    \vspace*{0.3cm}
    Im Nachfolgenden ist skizzenhaft dargestellt, welche Ausdehnung ausgewählte Graphalgorithmen erreichen, die dieses Problem lösen können.
    \begin{figure}[htb]
        \centering
        \begin{tikzpicture}[circ/.style={thick,circle,inner sep=0pt}]
            \node[] (a) at (0,0) {$\bullet_{s}$};
            \node[] (b) at (2,0) {$\bullet_{t}$};

            \node[circ, minimum size=2cm, draw=red] at (a)  {};
            \node[circ, minimum size=2cm, draw=red] at (b)  {};
            \node[circ, minimum size=3.85cm, draw=blue] at (a)  {};
            \draw[draw=green] (0.925,0.05) ellipse [x radius=1,y radius=0.5];

            \node (legend) [draw,align=left] at (5.5, 0) {\tikz\draw[blue, fill=blue] (0,0) circle (.5ex); Dijkstra\\
            \tikz\draw[green, fill=green] (0,-1) circle (.5ex); \text{A*} \\
            \tikz\draw[red, fill=red] (0,-1) circle (.5ex); Bidirectional};
        \end{tikzpicture}
    \end{figure}

    \section*{Contraction Hierarchies}\label{sec:contractionHierarchies}
    \subsection*{Anreiz}

    Für einen relativ statischen Graphen (wie das Straßennetz) ist es nicht sinnvoll und vor allem ineffizient,
    für jede Suchanfrage eines kürzesten Pfads einen Dijkstra-Algorithmus laufen zu lassen.
    Daher möchte man einen Vorverarbeitungsschritt einfügen, der anfangs einige Minuten Zeit benötigt,
    hinterher aber jede Anfrage enorm (um den Faktor $100.000$) beschleunigt.

    Ein naiver Ansatz hierfür wäre, alle paarweisen kürzesten Distanzen vorzuberechnen.
    Theoretisch wäre die Antwortzeit auf eine Anfrage damit sehr schnell, praktisch scheitert es aber daran,
    die Menge der paarweise kürzesten Pfade im Speicher zu halten.
    Für $n=20 \cdot 10^6$ Knoten ergäbe das etwa einen Platzverbrauch von $\approx 400 \cdot 10^{12}$.

    \subsection*{Idee: Vorverarbeitung}\label{subsec:idee:Vorverarbeitung}
    \begin{itemize}
        \item Der Graph wird um sogenannte \textit{Shortcuts} erweitert.
        Das sind zusätzliche Kanten, die \textit{kürzeste Wege} repräsentieren.
        \item Zusätzlich wird jedem Knoten ein \textit{Level} zugeordnet.
    \end{itemize}

    \subsection*{Idee: Anfrage}
    Eine Anfrage vom Startknoten $s$ zum Targetknoten $t$ folgt folgendem Muster:
    \begin{enumerate}
        \item Man lässt den Dijkstra-Algorithmus von $s$ laufen und berücksichtigt nur Kanten, die zu höherleveligen Knoten führen.
        \item Man lässt den Dijkstra-Algorithmus von $t$ laufen und berücksichtigt nur Kanten, die zu höherleveligen Knoten führen.
        \item Nun betrachtet man alle Knoten, die sowohl von $s$ als auch von $t$ besucht wurden. Die
        \textit{Kürzeste-Weg-Distanz} $d$ ist bestimmt durch das $v$, für das
        \begin{equation*}
            d_s(v) + d_t(v) = \text{minimal}
        \end{equation*}
    \end{enumerate}

    \subsection*{Vorverarbeitung: Knotenkontraktion}
    Die zentrale Operation der Contraction Hierarchies ist die Knotenkontraktion. Angenommen, ein kleiner Ausschnitt aus dem Straßengraphen hätte die folgende Struktur:
    \begin{figure}[htb]
        \centering
        \begin{tikzpicture}[myn/.style={circle,very thick,draw,inner sep=0.25cm,outer sep=0pt, fill=anti-flashwhite!100}]
            \node[myn] (a) at (4.5,3.5) {\textsf{a}};
            \node[myn] (b) at (5,1) {\textsf{b}};
            \node[myn] (c) at (3,-1) {\textsf{c}};
            \node[myn] (u) at (0,4) {\textsf{u}};
            \node[myn] (v) at (2,2) {\textsf{v}};
            \node[myn] (w) at (0,0) {\textsf{w}};


            \draw[-, very thick] (a) -- (v) node[midway, below] {12};
            \draw[-, very thick] (b) -- (v) node[midway, below] {9};
            \draw[-, very thick] (c) -- (v) node[midway,right] {1};
            \draw[-, very thick] (w) -- (v) node[midway, above] {5};
            \draw[-, very thick] (u) -- (v) node[midway, below] {6};
        \end{tikzpicture}
    \end{figure}

    \newpage
    Nun möchte man den Knoten $v$ entfernen, die Kanten zum Platz, an dem $v$ sich befand, aber erhalten.

    \begin{figure}[htb]
        \centering
        \begin{tikzpicture}[myn/.style={circle,very thick,draw,inner sep=0.25cm,outer sep=0pt, fill=anti-flashwhite!100}]
            \node[myn] (a) at (4.5,3.5) {\textsf{a}};
            \node[myn] (b) at (5,1) {\textsf{b}};
            \node[myn] (c) at (3,-1) {\textsf{c}};
            \node[myn] (u) at (0,4) {\textsf{u}};
            \coordinate  (v) at (2,2);
            \node[myn] (w) at (0,0) {\textsf{w}};


            \draw[-, very thick] (a) -- (v) node[midway, below] {12};
            \draw[-, very thick] (b) -- (v) node[midway, below] {9};
            \draw[-, very thick] (c) -- (v) node[midway,right] {1};
            \draw[-, very thick] (w) -- (v) node[midway, above] {5};
            \draw[-, very thick] (u) -- (v) node[midway, below] {6};
        \end{tikzpicture}
    \end{figure}

    Um jetzt die Kürzeste-Weg-Distanz zwischen zwei Knoten zu erhalten,
    muss der Pfad über \textit{ehemals $v$} gegangen werden.
    Die Idee ist nun, dass zwischen jedem Nachbarknotenpaar von $v$ eine neue Kante mit den Kosten \textit{Weg über $v$} eingefügt wird.
    Ausschnittsweise also

    \begin{figure}[htb]
        \centering
        \begin{tikzpicture}[myn/.style={circle,very thick,draw,inner sep=0.25cm,outer sep=0pt, fill=anti-flashwhite!100}]
            \node[myn] (a) at (4.5,3.5) {\textsf{a}};
            \node[myn] (b) at (5,1) {\textsf{b}};
            \node[myn] (c) at (3,-1) {\textsf{c}};
            \node[myn] (u) at (0,4) {\textsf{u}};
            \coordinate  (v) at (2,2);
            \node[myn] (w) at (0,0) {\textsf{w}};


            \draw[-, very thick] (a) -- (v) node[midway, below] {12};
            \draw[-, very thick] (b) -- (v) node[midway, below] {9};
            \draw[-, very thick] (c) -- (v) node[midway,right] {1};
            \draw[-, very thick] (w) -- (v) node[midway, above] {5};
            \draw[-, very thick] (u) -- (v) node[midway, below] {6};
            \draw [dashed, blue, very thick] (u) -- (w) node[midway, left] {11};
            \draw [dashed, blue, very thick] (u)  to [bend angle=45, bend left] node[midway, above] {18}(a) ;
            \draw [dashed, blue, very thick] (w)  to [bend angle=45, bend right, out=-70,in=-110, distance=3cm] node[midway, below] {14}(b) ;
        \end{tikzpicture}
    \end{figure}

    Nun kann aber der Fall auftreten, dass es schon einen Pfad zwischen einem Nachbarknotenpaar von $v$ gibt, der kürzer ist als der neu eingefügte Pfad über $v$.

    \begin{figure}[htb]
        \centering
        \begin{tikzpicture}[myn/.style={circle,very thick,draw,inner sep=0.25cm,outer sep=0pt, fill=anti-flashwhite!100}]
            \node[myn] (u) at (0,0.5) {\textsf{u}};
            \node[myn] (a) at (4.5,0) {\textsf{a}};
            \node (n) at (1.1,1.25) {\textcolor{red}{\(\not\)}};
            \node (n) at (3.68,0.75) {\textcolor{red}{\(\not\)}};


            \draw [dashed, blue, very thick] (u)  to [bend angle=45, bend left] node[midway, above] {18} node[midway, red]{\(\not\)}(a);
            \draw[-, very thick, gray] (u) -- (a) node[midway, below] {14};

        \end{tikzpicture}
    \end{figure}

    Diese falschen \textit{Shortcuts} werden daher nicht eingefügt.

    \vspace*{0.3cm}
    Das Verfahren lässt sich damit so zusammenfassen: \textbf{\textsf{Man fügt Shortcuts zwischen zwei Knoten $x$ und $y$ ein, genau dann wenn ein}} \textit{kürzester Weg} \textbf{\textsf{zwischen $x$ und $y$ existiert}}.

    \newpage
    \begin{lstlisting}[escapeinside={(;}{;)}]
        counter = 0
        while ((;$|V| > 1$;))
            Wähle ein (;$v \in V$;) und kontrahiere es
            Füge ein Shortcut zu (;$E$;) hinzu
            level[(;$v$;)] = counter++

        return level[] und alle kreierten Shortcuts
    \end{lstlisting}

    \subsection*{Vorverarbeitung: Korrektheit der Knotenkontraktion}
    Man betrachtet einen kürzesten Pfad von $s$ nach $t$, der die Knoten $v_1, v_2, \cdots, v_k$ passiert.
    \begin{figure}[H]
        \centering
        \begin{subfigure}{.9\textwidth}
            \centering
            \begin{tikzpicture}[thick, scale=0.8, myn/.style={circle,thick,draw,inner sep=0.1cm,outer sep=0pt, fill=anti-flashwhite!100}]
                \node[myn] (v2) at (3,0.5) {$v_2$};
                \node[myn, fill=amethyst] (t) at (13.5,1) {t};
                \node[myn, fill=applegreen] (s) at (0,1.5) {s};
                \node[myn] (v7) at (10.5,2) {$v_7$};
                \node[myn] (v3) at (4.5,2.5) {$v_3$};
                \node[myn] (v1) at (1.5,3) {$v_1$};
                \node[myn] (v5) at (7.5,3.5) {$v_5$};
                \node[myn] (v8) at (12,4) {$v_8$};
                \node[myn] (v4) at (6,4.5) {$v_4$};
                \node[myn] (v6) at (9,5) {$v_6$};

                \draw[-, thick] (s) -- (v1) node[midway, above left] {1};
                \draw[-, thick] (v1) -- (v2) node[midway, above right] {2};
                \draw[-, thick] (v2) -- (v3) node[midway, above left] {3};
                \draw[-, thick] (v3) -- (v4) node[midway, above left] {4};
                \draw[-, thick] (v4) -- (v5) node[midway, above] {5};
                \draw[-, thick] (v5) -- (v6) node[midway, above left] {4};
                \draw[-, thick] (v6) -- (v7) node[midway, above right] {3};
                \draw[-, thick] (v7) -- (v8) node[midway, above left ] {2};
                \draw[-, thick] (v8) -- (t) node[midway, above right] {1};
            \end{tikzpicture}
        \end{subfigure}%
        \begin{subfigure}{.1\textwidth}
            \centering
            \begin{tikzpicture}[thick, scale=0.8]
                \coordinate  (start) at (0,0);
                \coordinate  (end) at (0,5);
                \draw[->, very thick] (start) -- (end) node[midway, right] {Level};
            \end{tikzpicture}
        \end{subfigure}
    \end{figure}

    Die Verarbeitungsreihenfolge der Knoten entspricht der Ordnung der Knotenlevel:
    \begin{equation*}
        v_2 \rightarrow t \rightarrow s \rightarrow v_7 \rightarrow v_3 \rightarrow v_1 \rightarrow v_5 \rightarrow v_8 \rightarrow v_4 \rightarrow v_6
    \end{equation*}

    Als Erstes wird also der Knoten $v_2$ kontrahiert. Das ergibt
    \begin{figure}[H]
        \centering
        \begin{subfigure}{.9\textwidth}
            \centering
            \begin{tikzpicture}[thick, scale=0.8, myn/.style={circle,thick,draw,inner sep=0.1cm,outer sep=0pt, fill=anti-flashwhite!100}]
                \node[myn] (v2) at (3,0.5) {$v_2$};
                \node[myn, fill=amethyst] (t) at (13.5,1) {t};
                \node[myn, fill=applegreen] (s) at (0,1.5) {s};
                \node[myn] (v7) at (10.5,2) {$v_7$};
                \node[myn] (v3) at (4.5,2.5) {$v_3$};
                \node[myn] (v1) at (1.5,3) {$v_1$};
                \node[myn] (v5) at (7.5,3.5) {$v_5$};
                \node[myn] (v8) at (12,4) {$v_8$};
                \node[myn] (v4) at (6,4.5) {$v_4$};
                \node[myn] (v6) at (9,5) {$v_6$};

                \draw[-, thick] (s) -- (v1) node[midway, above left] {1};
                \draw[-, thick] (v1) -- (v2) node[midway, above right] {2};
                \draw[-, thick] (v2) -- (v3) node[midway, above left] {3};
                \draw[-, thick] (v3) -- (v4) node[midway, above left] {4};
                \draw[-, thick] (v4) -- (v5) node[midway, above] {5};
                \draw[-, thick] (v5) -- (v6) node[midway, above left] {4};
                \draw[-, thick] (v6) -- (v7) node[midway, above right] {3};
                \draw[-, thick] (v7) -- (v8) node[midway, above left ] {2};
                \draw[-, thick] (v8) -- (t) node[midway, above right] {1};
                \draw [dashed, red, very thick] (v1) -- (v3) node[midway, above] {5};
                %                \draw[-, thick] (s) -- (v1);
            \end{tikzpicture}
        \end{subfigure}%
        \begin{subfigure}{.1\textwidth}
            \centering
            \begin{tikzpicture}[thick, scale=0.8]
                \coordinate  (start) at (0,0);
                \coordinate  (end) at (0,5);
                \draw[->, very thick] (start) -- (end) node[midway, right] {Level};
            \end{tikzpicture}
        \end{subfigure}
    \end{figure}

    Danach werden die Knoten $s$ und $t$ kontrahiert. Da beide Randknoten sind, ändert sich jedoch nichts.
    Es folgt $v_7$,  $v_3$ usw.

    \vspace*{0.3cm}
    Am Ende ergibt das folgendes Bild:

    \begin{figure}[H]
        \centering
        \begin{subfigure}{.9\textwidth}
            \centering
            \begin{tikzpicture}[thick, scale=0.8, myn/.style={circle,thick,draw,inner sep=0.1cm,outer sep=0pt, fill=anti-flashwhite!100}]
                \node[myn] (v2) at (3,0.5) {$v_2$};
                \node[myn, fill=amethyst] (t) at (13.5,1) {t};
                \node[myn, fill=applegreen] (s) at (0,1.5) {s};
                \node[myn] (v7) at (10.5,2) {$v_7$};
                \node[myn] (v3) at (4.5,2.5) {$v_3$};
                \node[myn] (v1) at (1.5,3) {$v_1$};
                \node[myn] (v5) at (7.5,3.5) {$v_5$};
                \node[myn] (v8) at (12,4) {$v_8$};
                \node[myn] (v4) at (6,4.5) {$v_4$};
                \node[myn] (v6) at (9,5) {$v_6$};

                \draw[-, thick] (s) -- (v1) node[midway, above left] {1};
                \draw[-, thick] (v1) -- (v2) node[midway, above right] {2};
                \draw[-, thick] (v2) -- (v3) node[midway, above left] {3};
                \draw[-, thick] (v3) -- (v4) node[midway, above left] {4};
                \draw[-, thick] (v4) -- (v5) node[midway, above] {5};
                \draw[-, thick] (v5) -- (v6) node[midway, above left] {4};
                \draw[-, thick] (v6) -- (v7) node[midway, above right] {3};
                \draw[-, thick] (v7) -- (v8) node[midway, above left ] {2};
                \draw[-, thick] (v8) -- (t) node[midway, above right] {1};
                \draw [dashed, red, thick] (v1) -- (v3) node[midway, above] {5};
                \draw [dashed, red, thick] (v1) -- (v4) node[midway, above] {9};
                \draw [dashed, red, thick] (v1) -- (v4) node[midway, above] {9};
                \draw [dashed, red, thick] (v4) -- (v6) node[midway, above] {9};
                \draw [dashed, red, thick] (v6) -- (v8) node[midway, above] {5};
                %                \draw[-, thick] (s) -- (v1);
            \end{tikzpicture}
        \end{subfigure}%
        \begin{subfigure}{.1\textwidth}
            \centering
            \begin{tikzpicture}[thick, scale=0.8]
                \coordinate  (start) at (0,0);
                \coordinate  (end) at (0,5);
                \draw[->, very thick] (start) -- (end) node[midway, right] {Level};
            \end{tikzpicture}
        \end{subfigure}
    \end{figure}
    \proofend

    \subsection*{Anmerkungen zur Kontraktionsreihenfolge}
    Man möchte möglichst wenige Shortcuts neu einfügen.
    Es gibt daher verschiedene Ansätze, die Anzahl der neu eingefügten Shortcuts durch die Kontraktionsreihenfolge zu beeinflussen.

    \subsubsection*{Ansatz 1: Edge-Difference}
    Man berechnet für jeden Knoten die sogenannte \textit{Edge Difference}:
    \begin{equation*}
        \underbrace{\text{Edge Difference}}_{\tikz\draw[orange, fill=orange] (0,0) circle (.5ex); \phantom{ }} = \text{Anzahl neu einzufügender Shortcuts} - \text{Anzahl wegfallender Kanten}
    \end{equation*}

    \begin{figure}[H]
        \centering
        \begin{tikzpicture}[thick, scale=1,
        mynab/.style={circle,thick,draw,inner sep=2pt,outer sep=0pt, fill=anti-flashwhite!100, label=above:\small\textcolor{orange}{#1}\strut},
        mynbe/.style={circle,thick,draw,inner sep=2pt,outer sep=0pt, fill=anti-flashwhite!100, label=below:\small\textcolor{orange}{#1}\strut},
        mynal/.style={circle,thick,draw,inner sep=2pt,outer sep=0pt, fill=anti-flashwhite!100, label=above left:\small\textcolor{orange}{#1}\strut},
        ]
            \node[mynab=-1] (v1) at (-4,-0.5) {};
            \node[mynab=-1] (v2) at (-2,0.5) {};
            \node[mynab=0] (v3) at (0,-0.5) {};
            \node[mynab=-1] (v4) at (2,0.5) {};
            \node[mynab=-1] (v5) at (4,-0.5) {};

            \node[mynal=2] (v6) at (0,-2.5) {};
            \node[mynab=-1] (v7) at (-2,-3) {};
            \node[mynab=-1] (v8) at (2,-3) {};

            \node[mynbe=-1] (v9) at (0,-5) {};

            \draw[-,thick] (v1) -- (v2) node[midway, above] {\tiny1};
            \draw[-,thick] (v2) -- (v3) node[midway, above] {\tiny1};
            \draw[-,thick] (v3) -- (v4) node[midway, above] {\tiny1};
            \draw[-,thick] (v4) -- (v5) node[midway, above] {\tiny1};

            \draw[-,thick] (v3) -- (v6) node[midway, left] {\tiny1};
            \draw[-,thick] (v6) -- (v7) node[midway, above] {\tiny1};
            \draw[-,thick] (v6) -- (v8) node[midway, above] {\tiny1};

            \draw[-,thick] (v6) -- (v9) node[midway, left] {\tiny1};

        \end{tikzpicture}
    \end{figure}%

    Nun kontrahiert man immer einen Knoten mit der minimalen \textit{Edge Difference}.

    \subsubsection*{Ansatz 2: Jeden zweiten Knoten kontrahieren}

    \begin{figure}[H]
        \centering
        \begin{tikzpicture}[thick, scale=1, myn/.style={circle,thick,draw,inner sep=2pt,outer sep=0pt, fill=anti-flashwhite!100}, ]
            \node[myn] (v0) at (-6,0) {};
            \node[myn] (v1) at (-4,0) {};
            \node[myn] (v2) at (-2,0) {};
            \node[myn] (v3) at (0,-0) {};
            \node[myn] (v4) at (2,0) {};
            \node[myn] (v5) at (4,0) {};
            \node[myn] (v6) at (6,0) {};

            \draw[-,thick] (v0) -- (v1) node[midway, above] {\tiny1};
            \draw[-,thick] (v1) -- (v2) node[midway, above] {\tiny1};
            \draw[-,thick] (v2) -- (v3) node[midway, above] {\tiny1};
            \draw[-,thick] (v3) -- (v4) node[midway, above] {\tiny1};
            \draw[-,thick] (v4) -- (v5) node[midway, above] {\tiny1};
            \draw[-,thick] (v5) -- (v6) node[midway, above] {\tiny1};

            \draw [-,anchor=center, very thick, draw=red](v1) ellipse (0.3cm and 0.6cm);
            \draw [-,anchor=center, very thick, draw=red](v3) ellipse (0.3cm and 0.6cm);
            \draw [-,anchor=center, very thick, draw=red](v5) ellipse (0.3cm and 0.6cm);

        \end{tikzpicture}
    \end{figure}

    \subsubsection*{Ansatz 3: Zufällige Reihenfolge}
    Man kann die Knoten auch in zufälliger Reihenfolge kontrahieren.

    \subsubsection*{Ansatz 4: In der Praxis sehr gut}
    Man erlaubt Knoten, die nicht benachbart sind, dasselbe Level.
    Eine \textit{Kontraktionsrunde} läuft dann nach folgendem Schema ab:

    \begin{enumerate}
        \item Man berechnet eine unabhängige Menge $I$ (also eine Teilmenge von Knoten, die nicht zueinander benachbart sind) von $G$.
        \item Nun berechnet man wieder die \textit{Edge Differences} für alle $v \in I$.
        \item Man kontrahiert alle $v \in I$ mit den $75\, \%$ kleinsten \textit{Edge Differences}.
        \item Alle Knoten in dieser Runde erhalten dasselbe Level.
    \end{enumerate}

    \subsection*{Anmerkungenen und Notizen}\label{subsec:notizenZurContractionHierarchie}
    \begin{itemize}
        \item Für Deutschland benötigt die Vorverarbeitung etwa $4$ Minuten.
        \item Am Anfang laufen viele kleine Dijkstra-Algorithmen, später wenige längere.
        \item Die Kontraktionsstrategien eliminieren in aller Regel zuerst kleine Dörfer und Dorfstraßen,
        danach Landstraßen, Bundesstraßen und erst zum Schluss Autobahnen. Die Strategie liefert also eine Art \textbf{\textsf{Level of Detail}} gleich mit.
        \item Der ermittelte kürzeste Pfad enthält Shortcuts, die sehr einfach \textit{entpackt} werden können.
        \item Anfragen benötigen nach der Vorverarbeitung etwa \SI{5}{\ms}. Diese Zeit kann durch diverse Tricks auf $<$ \SI{1}{\ms} reduziert werden.
        \item Es gibt noch einige andere sehr schnelle Algorithmen für das Ermitteln des kürzesten Pfads in einem Graphen mit Vorverarbeitung:
        \begin{itemize}
            \item \textsf{Transit Nodes} $\quad$ benötigt etwa \SI{10}{\micro\s} für eine Anfrage
            \item \textsf{Hublabels} $\quad$ benötigt etwa \SI{10}{\micro\s} für eine Anfrage
            \item \textsf{CAP} $\quad$ benötigt etwa \SI{1}{\ms} für eine Anfrage
        \end{itemize}
        \item Für den Öffentlichen Verkehr gibt es spezielle Anforderungen. Hier arbeitet man mit Transit Patterns.
    \end{itemize}


\end{document}