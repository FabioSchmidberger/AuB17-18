\documentclass{scrartcl}% siehe <http://www.komascript.de>
\usepackage[margin=2.5cm]{geometry}
\usepackage[utf8]{inputenc}
\usepackage[T1]{fontenc}
\usepackage[ngerman]{babel}
\usepackage{lmodern}
\usepackage{amsmath}
\usepackage{amssymb}
\usepackage{cancel}
\usepackage{hyperref}
\usepackage{mathrsfs}
\usepackage{mathtools}
\usepackage{listings}

\lstset{basicstyle=\ttfamily}
\lstset{literate=%
{Ö}{{\"O}}1
{Ä}{{\"A}}1
{Ü}{{\"U}}1
{ß}{{\ss}}1
{ü}{{\"u}}1
{ä}{{\"a}}1
{ö}{{\"o}}1
}

\newtheorem{theorem}{Aussage}

\newcommand{\includepic}[2]{\includegraphics[width=#2\textwidth,height=#2\textheight,keepaspectratio]{#1}}

\usepackage[yyyymmdd,hhmm]{datetime}

% Base path of images
\usepackage{graphicx}
\graphicspath{ {../lectures-img/} }

\begin{document}
    % ----------------------------------------------------------------------------
    \subject{Vorlesungsmitschrieb}
    \title{Algorithmen und Berechenbarkeit}
    \subtitle{Vorlesung 03}
    \date{Letztes Update: \today \ - \currenttime \ Uhr}
    \maketitle
    % ----------------------------------------------------------------------------

    \subsection*{Laufzeiten}
    \label{subsec:laufzeiten}

    \texttt{Es wäre gut, wenn jemand die Inhalte der ersten 15 Minuten nachtragen könnte, da ich leider verhindert war.}

    \begin{itemize}
        \item $O(n^2)$ beschreibt die Menge aller Funktionen, die echt langsamer wachsen als $n^2$.
        \begin{itemize}
            \item \textbf{In $O(n^2)$:}\quad $n,\quad n^{1,99},\quad n \cdot \log(n),\quad n \cdot \log^2(n)$
            \item \textbf{Nicht in $O(n^2)$:}\quad $n^2,\quad n^2 \cdot \log(n),\quad 2^n$
        \end{itemize}
        \item $\Omega(n^2)$ beschreibt die Menge aller Funktionen, die mindestens so schnell wie $n^2$ wachsen (asymptotisch).
        \begin{itemize}
            \item \textbf{In $\Omega(n^2)$:}\quad $n^2,\quad n^2 \cdot \log(n),\quad n^{2,1},\quad 2^n$
            \item \textbf{Nicht in $\Omega(n^2)$:}\quad $n \cdot \log(n),\quad n^{1,9},\quad \sqrt{n},\quad n \cdot \log^2(n),\quad \frac{n^2}{\log(n)}$
        \end{itemize}
        \item $\omega(n^2)$ beschreibt die Menge aller Funktionen, die echt schneller wachsen als $n^2$.
        \begin{itemize}
            \item \textbf{In $\omega(n^2)$:}\quad $n^{2,1},\quad n^2 \cdot \log(n), \quad 2^n$
            \item \textbf{Nicht in $\omega(n^2)$:}\quad $n^2, \quad n \cdot \log(n), \quad \frac{n^2}{\log(n)}, \quad \log(n)$
        \end{itemize}
        \item $\Theta(n^2)$ beschreibt die Menge aller Funktionen, die sowohl in $O(n^2)$ als auch in $\Omega(n^2)$ enthalten sind.
    \end{itemize}

    \subsection*{Vergleichsbasiertes Sortieren}
    \label{subsec:vergleichsbasiertessortieren}

    Es wird \textit{Vergleichsbasiertes Sortieren} von $n$ Objekten betrachtet, wobei die Elemente nur verglichen werden dürfen.
    Hier stellt sich die \textbf{Frage}: Was ist die Komplexität des vergleichsbasierten Sortierens?

    Die ersten Überlegungen stellen klar: Jeder Sortieralgorithmus liefert eine obere Schranke für $T(n)$:
    \begin{itemize}
        \item \textbf{Bubblesort} $\Rightarrow T(n) \in O(n^2)$
        \item \textbf{Mergesort} $\Rightarrow T(n) \in O(n \cdot \log(n))$
    \end{itemize}
    \newpage
    \begin{theorem}
        Es kann bewiesen werden, dass $T(n) \in \Omega(n \cdot \log(n)) \quad \Rightarrow \quad T(n) \in \Theta(n \cdot \log(n))$
    \end{theorem}

    \begin{proof}[\textbf{Beweis}]
        Man betrachte einen beliebigen Algorithmus $\mathcal{A}$ zum Sortieren. $\mathcal{A}$ vergleicht $e_i$ mit $e_j$.

        \begin{figure}[htb]
            \centering
            \includepic{lec_03_a}{0.3}
            \caption{Exemplarische Vergleiche beim vergleichsbasierten Sortieren}
        \end{figure}

        Ein Blatt in diesem Baum heißt im Prinzip: Der Algorithmus hat fertig sortiert.
        Es heißt aber genauso: Der Algorithmus hat "`herausgefunden"', was die "`Permutation"' der Eingabe war.
        Daraus folgt, dass der Baum $n!$ Blätter haben muss.
        Es zeigt sich außerdem, dass die Worst-Case-Laufzeit des Algorithmus $\mathcal{A}$ genau der Tiefe des Baumes entspricht.
        Damit stellt sich die nächste \textbf{Frage}:

        Was ist die minimale Tiefe eines Binärbaumes, der $n!$ Blätter hat?

        \begin{equation*}
            2^n = n!
        \end{equation*}
        \begin{equation*}
            \left(n! \approx \left({\frac{n}{e}}\right)^{\frac{n}{e}} \Rightarrow x = \log_2\left(\frac{n}{e}\right)^{\frac{n}{e}}
            \Rightarrow x = n \cdot \log(n)\right)
        \end{equation*}

    \end{proof}

    \subsection*{Monte-Carlo und Las-Vegas Algorithmen ineinander umwandeln}
    \label{subsec:monte-carloUndLas-vegasAlgorithmenIneinanderUmwandeln}

    Es stellt sich die \textbf{Frage}, ob jeder Las-Vegas-Algorithmus in einen Monte-Carlo-Algorithmus umgewandelt werden kann, und andersherum.

    \subsubsection*{Las-Vegas $\Rightarrow$ Monte-Carlo}
    Die Idee besteht aus folgender Überlegung: Lasse den Las-Vegas-Algorithmus eine bestimmte Anzahl an Schritten laufen und breche dann ab.
    War der Las-Vegas-Algorithmus bis dahin fertig, so muss auch das Ergebnis korrekt sein.
    War der Las-Vegas-Algorithmus bis dahin nicht fertig, dann gibt es auch kein korrektes Ergebnis.

    Die zentrale \textbf{Frage}, die sich hier anfügt: Wie lange darf der Algorithmus laufen und was ist seine Erfolgswahrscheinlichkeit?

    Sei nun $\mathcal{A}$ ein Las-Vegas-Algorithmus mit erwarteter Laufzeit $f(n)$.
    $\mathcal{A}$ darf nun für maximal $\alpha \cdot f(n)\ | \alpha \geq 1$ Schritte laufen.
    Falls der Algorithmus bis dahin fertig ist, ist das Ergebnis sicher korrekt.
    Ist der Algorithmus bis dahin nicht fertig, so wird \textit{Müll} zurückgegeben.

    Dieser modifizierte Algorithmus hat immer eine Laufzeit von $< \alpha \cdot f(n)$.
    Die Wahrscheinlichkeit, dass \textit{Müll} zurückgegeben wird, ist gleich der Wahrscheinlichkeit,
    dass $\mathcal{A}$ länger als $\alpha \cdot f(n)$ Zeit zum Sortieren benötigt.

    \subsubsection*{Monte-Carlo $\Rightarrow$ Las-Vegas}

    Nicht alle Monte-Carlo-Algorithmen können ohne Weiteres in Las-Vegas-Algorithmen umgewandelt werden.

    Für manche Probleme ist die Verifikation des Ergebnisses einfacher als die Berechnung:
    \begin{itemize}
        \item Sortieren: $O(n \cdot \log(n))$ \quad vs. \quad $O(n)$.
        \item Kürzeste Wege: $O(n \cdot \log(n+m))$ \quad vs. \quad $O(m)$.
    \end{itemize}

    Das Überführen von Monte-Carlo- in Las-Vegas-Algorithmen ist möglich, wenn man einen effizienten \textit{Checker} hat.

    \begin{itemize}
        \item Monte-Carlo-Algorithmus $\mathcal{A}$ hat eine Laufzeit von $f(n)$.
        \item Checker $\mathcal{O}$ hat eine Laufzeit von $f(n)$.
        \item Die Erfolgswahrscheinlichkeit von $\mathcal{A}$ sei $p(n)$.
    \end{itemize}

    Mit folgendem Algorithmus kann dann ein Las-Vegas-Algorithmus erzeugt werden:

    \begin{lstlisting}
        1. Lasse Algorithmus laufen
        2. Überprüfe Ergebnis mit Checker
            Falls korrekt, dann fertig
            Falls nicht korrekt, zurück zu 1.
    \end{lstlisting}

    Die erwartete Laufzeit $\mathcal{R}$ ist dann

    \begin{equation*}
        \begin{flalign}
            \mathcal{R} &= p(n) \cdot (f(n) + g(n))&&\\\nonumber
                        &\quad + ((1 - p(n)) \cdot p(n) \cdot (f(n) + g(n))) \cdot 2&&\\\nonumber
                        &\quad + ((1 - p(n))^2 \cdot p(n) \cdot (f(n) + g(n))) \cdot 3&&\\\nonumber
                        &\quad +\ ...&&\\\nonumber
                        &=\left( f(n) + g(n)) \cdot p(n) \cdot \sum^{\infty}_{i=1}((1-p(n))^i < \frac{f(n) + g(n)}{p(n)}) \cdot (i+1)\right)
        \end{flalign}
    \end{equation*}

    \hrulefill

    \section*{Anhang}
    \label{sec:anhang}

    \subsubsection*{Markov-Ungleichung}
    Sei $X$ eine nicht negative Zufallsvariable mit $E[X] = \mu$.
    Dann gilt:

    \begin{equation*}
        P(X \geq \alpha \cdot \mu ) \leq \frac{1}{\alpha}
    \end{equation*}

\end{document}