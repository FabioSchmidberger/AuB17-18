\documentclass{scrartcl}% siehe <http://www.komascript.de>
% packages ---------------------------------------------------------------------------------------------------- packages
\usepackage[margin=2.5cm]{geometry}
\usepackage[utf8]{inputenc}
\usepackage[T1]{fontenc}
\usepackage[ngerman]{babel}
\usepackage{lmodern}
\usepackage{amsmath}
\usepackage{amssymb}
\usepackage{amsfonts}
\usepackage{cancel}
\usepackage{listings}
\usepackage{multirow}
\usepackage{hhline}
\usepackage{tikz}
\usepackage{pgfplots}
\usepackage[all,dvips,arc,curve,color,frame]{xy}
\usepackage[yyyymmdd,hhmm]{datetime}
\usepackage{float}
\usepackage{hyperref}
\usepackage[table]{xcolor,colortbl}
\usepackage{graphicx}
\usepackage{subcaption}
\usepackage{tabularx}
\usepackage{footnote}
\usepackage{multicol}
\usepackage{stackengine}
\usepackage[norndcorners,customcolors,shade]{hf-tikz}
\usepackage{siunitx}
\usepackage{algorithm}
\usepackage[]{algorithmic}
% \usepackage[noend]{algorithmic}

% pgf plotsets -------------------------------------------------------------------------------------------- pgf plotsets
\pgfplotsset{ticks=none}
\pgfplotsset{width=10cm, compat=1.5.1}

% tikz libraries ---------------------------------------------------------------------------------------- tikz libraries
\usetikzlibrary{shapes}
\usetikzlibrary{positioning}
\usetikzlibrary{shapes.misc}
\usetikzlibrary{shapes.arrows,calc,quotes,babel}
\usetikzlibrary{positioning,arrows,chains,scopes,fit}
\usetikzlibrary{matrix,backgrounds}
\usetikzlibrary{trees}
\usetikzlibrary{calc}
\usetikzlibrary{tikzmark}

\sisetup{
locale = DE ,
per-mode = symbol
}

\algsetup{indent=2em}
\newlength\myindent
\setlength\myindent{2em}
\newcommand\bindent{%
    \begingroup
    \setlength{\itemindent}{\myindent}
    \addtolength{\algorithmicindent}{\myindent}
}
\newcommand\eindent{\endgroup}

% commands ---------------------------------------------------------------------------------------------------- commands
\setlength{\parindent}{0pt}
\newcommand{\includepic}[2]{\includegraphics[width=#2\textwidth,height=#2\textheight,keepaspectratio]{#1}}
\newcommand\circlearound[1]{%
\tikz[baseline]\node[draw,shape=circle,anchor=base] {#1} ;}
\newcommand{\emptyrow}[0]{\multicolumn{1}{c}{} & \multicolumn{1}{c}{} & \multicolumn{1}{c}{} & \multicolumn{1}{c}{} &
\multicolumn{1}{c}{} & \multicolumn{1}{c}{} & \multicolumn{1}{c}{} & \multicolumn{1}{c}{} &
\multicolumn{1}{c}{} & \multicolumn{1}{c|}{} & & \\}
\newcommand\underbracetext[1]{\raisebox{1ex}{\ensuremath{\underbrace{\hphantom{\text{#1}}}_{\text{\normalsize #1}}}}}
\newcommand{\headerline}[3]{
\subject{#2}
\title{#1}
\subtitle{#3}
\date{Letztes Update: \today \ - \currenttime \ Uhr}
\maketitle
}
\newcommand{\newproof}[2]{
\vspace*{0.3cm} \textbf{\textsf{Satz:}} #1

\vspace*{0.3cm} \textbf{\textsf{Beweis:}} #2 \vspace*{0.3cm}
}
\newcommand{\uparrow}[0]{\rlap{\scalebox{1.6}{$\uparrow$}}}
\newlength\dlf
\newcommand{\proofend}[0]{\hfill $\square$}
\newcommand\drawbi[7][4pt]{%
    \path(#2)--(#3)coordinate[at start](h1)coordinate[at end](h2);
    \draw[#4]($(h1)!#1!90:(h2)$)-- node [auto=left] {#5} ($(h2)!#1!-90:(h1)$);
    \draw[#6]($(h1)!#1!-90:(h2)$)-- node [auto=right] {#7} ($(h2)!#1!90:(h1)$);
    }

% colors -------------------------------------------------------------------------------------------------------- colors
\definecolor{amethyst}{rgb}{0.6, 0.4, 0.8}
\definecolor{anti-flashwhite}{rgb}{0.95, 0.95, 0.96}
\definecolor{applegreen}{rgb}{0.55, 0.71, 0.0}
\definecolor{babyblue}{rgb}{0.54, 0.81, 0.94}
\definecolor{bistre}{rgb}{0.24, 0.17, 0.12}
\definecolor{buff}{rgb}{0.94, 0.86, 0.51}
\definecolor{dodgerblue}{rgb}{0.12, 0.56, 1.0}
\definecolor{icterine}{rgb}{0.99, 0.97, 0.37}
\definecolor{tan}{rgb}{0.82, 0.71, 0.55}
\definecolor{upforestgreen}{rgb}{0.0, 0.27, 0.13}

% listing colors
\definecolor{listingBlue}{rgb}{0.13,0.13,1}
\definecolor{listingGreen}{rgb}{0,0.5,0}
\definecolor{listingRed}{rgb}{0.9,0,0}
\definecolor{listingGrey}{rgb}{0.46,0.45,0.48}

% listings ---------------------------------------------------------------------------------------------------- listings
\lstset{basicstyle=\ttfamily}
\lstset{literate=%
{Ö}{{\"O}}1
{Ä}{{\"A}}1
{Ü}{{\"U}}1
{ß}{{\ss}}1
{ü}{{\"u}}1
{ä}{{\"a}}1
{ö}{{\"o}}1
}
\lstset{language=Java,
showspaces=false,
showtabs=false,
breaklines=true,
showstringspaces=false,
breakatwhitespace=true,
commentstyle=\color{listingGreen},
keywordstyle=\color{listingBlue},
stringstyle=\color{listingRed},
basicstyle=\ttfamily,
moredelim=[il][\textcolor{listingGrey}]{$$},
moredelim=[is][\textcolor{listingGrey}]{\%\%}{\%\%}
}
\lstdefinelanguage{Kotlin}{
  keywords={package, as, typealias, this, super, val, var, fun, for, null, true, false, is, in, throw, return, break, continue, object, if, try, else, while, do, when, yield, typeof, yield, typeof, class, interface, enum, object, override, public, private, get, set, import, abstract, },
  keywordstyle=\color{listingBlue}\bfseries,
  ndkeywords={@Deprecated, Iterable, Int, Integer, Float, Double, String, Runnable, dynamic},
  ndkeywordstyle=\color{amethyst}\bfseries,
  emph={println, return@, forEach,},
  emphstyle={\color{orange}},
  identifierstyle=\color{black},
  sensitive=true,
  commentstyle=\color{listingGreen}\ttfamily,
  comment=[l]{//},
  morecomment=[s]{/*}{*/},
  stringstyle=\color{listingRed}\ttfamily,
  morestring=[b]",
  morestring=[s]{"""*}{*"""},
  numbers=left,
}

% misc ------------------------------------------------------------------------------------------------------------ misc
\graphicspath{ {../lectures-img/} }
\setlength{\parindent}{0pt}

\newcolumntype{C}[1]{>{\centering\arraybackslash}p{#1}} % zentriert mit Breitenangabe
\newcolumntype{L}[1]{>{\raggedright\arraybackslash}p{#1}} % rechtsbündig mit Breitenangabe
\newcolumntype{R}[1]{>{\raggedleft\arraybackslash}p{#1}} % rechtsbündig mit Breitenangabe


\begin{document}
    \headerline{Algorithmen und Berechenbarkeit}{Vorlesungsmitschrift}{Vorlesung 09}

    \subsection*{Ansatz 2: Zweistufiges perfektes Hashing}
    Die Idee für ein mehrstufiges perfektes Hashing ist im Folgenden aufgezeichnet:

    \begin{figure}[H]
        \centering
        \begin{tikzpicture}[thick, scale=1]
            \node[node draw=none] (n00) at (-5,2.5) {$S \text{ mit } x $};
            \draw[|->, thin] (-4,2.5) -- (-1.5,2.5) node [midway, sloped, above] {$h$};

            % Primäry table
            % Coordinates
            \coordinate (c01) at (0,0);
            \coordinate (c02) at (0,0.5);
            \coordinate (c03) at (0,1);
            \coordinate (c04) at (0,1.5);
            \coordinate (c05) at (0,2);
            \coordinate (c06) at (0,2.5);
            \coordinate (c07) at (0,3);
            \coordinate (c08) at (0,3.5);
            \coordinate (c09) at (0,4);
            \coordinate (c10) at (0,4.5);
            \coordinate (c11) at (0,5);
            \coordinate (c12) at (3,5);
            \coordinate (c13) at (3,4.5);
            \coordinate (c14) at (3,4);
            \coordinate (c15) at (3,3.5);
            \coordinate (c16) at (3,3);
            \coordinate (c17) at (3,2.5);
            \coordinate (c18) at (3,2);
            \coordinate (c19) at (3,1.5);
            \coordinate (c20) at (3,1);
            \coordinate (c21) at (3,0.5);
            \coordinate (c22) at (3,0);

            \fill [fill=tan!20]
            plot [] coordinates {
            (c05) (c06) (c17) (c18)
            }
            plot [] coordinates {
            (c08) (c09) (c14) (c15)
            }
            ;

            \foreach \from/\to in {c01/c11, c11/c12, c12/c22, c22/c01, c05/c18, c06/c17, c09/c14, c08/c15}
            \draw[thin] (\from) -- (\to);

            \foreach \from/\to in {c02/c21, c03/c20, c04/c19, c07/c16, c10/c13}
            \draw[thin, dashed] (\from) -- (\to);

            \node[node draw=none] (n01) at (0.5,2.15) {$\cdots$};
            \node[node draw=none] (n02) at (0.5,3.65) {$\cdots$};
            \node[node draw=none] (n03) at (-0.2,4.65) {\tiny $0$};
            \node[node draw=none] (n04) at (-0.2,4.15) {\tiny $1$};
            \node[node draw=none] (n05) at (-0.2,3.65) {\tiny $2$};
            \node[node draw=none] (n06) at (-0.2,3.15) {\tiny $\vdots$};
            \node[node draw=none] (n07) at (-0.2,2.65) {};
            \node[node draw=none] (n08) at (-0.2,2.15) {\tiny $i$};
            \node[node draw=none] (n09) at (-0.2,1.65) {};
            \node[node draw=none] (n10) at (-0.2,1.15) {\tiny $\vdots$};
            \node[node draw=none] (n11) at (-0.2,0.65) {};
            \node[node draw=none] (n12) at (-0.5,0.15) {\tiny $m-1$};

            % Secondary table
            % top
            \coordinate (c30) at (5,3.5);
            \coordinate (c31) at (5,3.75);
            \coordinate (c32) at (5,4);
            \coordinate (c33) at (5,4.25);
            \coordinate (c34) at (5,4.5);
            \coordinate (c35) at (6.5,4.5);
            \coordinate (c36) at (6.5,4.25);
            \coordinate (c37) at (6.5,4);
            \coordinate (c38) at (6.5,3.75);
            \coordinate (c39) at (6.5,3.5);

            \foreach \from/\to in {c30/c34, c34/c35, c35/c39, c39/c30}
            \draw[thin] (\from) -- (\to);

            \foreach \from/\to in {c31/c38, c32/c37, c33/c36}
            \draw[ultra thin, dashed] (\from) -- (\to);

            % bottom

            \coordinate (c40) at (5,1.5);
            \coordinate (c41) at (5,1.75);
            \coordinate (c42) at (5,2);
            \coordinate (c43) at (5,2.25);
            \coordinate (c44) at (5,2.5);
            \coordinate (c45) at (6.5,2.5);
            \coordinate (c46) at (6.5,2.25);
            \coordinate (c47) at (6.5,2);
            \coordinate (c48) at (6.5,1.75);
            \coordinate (c49) at (6.5,1.5);

            \foreach \from/\to in {c40/c44, c44/c45, c45/c49, c49/c40}
            \draw[thin] (\from) -- (\to);

            \foreach \from/\to in {c41/c48, c42/c47, c43/c46}
            \draw[ultra thin, dashed] (\from) -- (\to);

            % arrows between hash tables
            \draw[|->, thin] (3.1,3.75) -- (4.9,4) node [midway, sloped, above] {};
            \draw[|->, thin] (3.1,2.25) -- (4.9,2) node [midway, sloped, above] {$h_i$};

            % Descriptions
            \node[node draw=none] (n20) at (1.5,-0.5) {$\substack{\text{Primärhashtafel der} \\ \text{Größe } \mathcal{O}(n)}$  };
            \node[node draw=none] (n20) at (5.75,-0.5) {$\substack{\text{Sekundär-} \\ \text{hashtafeln}}$ };

        \end{tikzpicture}
    \end{figure}

    Man wählt eine Primärhashfunktion $h$, sodass etwa linear viele Kollisionen erzeugt werden, also $c_S(h) = \mathcal{O}(n)$.
    Zusätzlich erzeugt man für jeden Primärhashtafeleintrag $i$,
    für den es mehr als einen Eintrag gibt, eine Sekundärhashfunktion $h_i$, die injektiv in eine Sekundärhashtafel hasht.

    \vspace*{0.3cm}
    Des Weiteren werden festgelegt:
    \begin{equation*}
        \begin{align*}
            m &= \text{Größe der Primärhashtafel} \\\nonumber
            m_i &= \text{Größe der $i$-ten Sekundärhashtafel} \\\nonumber
            l_i &= \text{Anzahl der Elemente, die von der Primärhashfunktion $h$ auf $i$ gemappt wurden.} \\\nonumber
        \end{align*}
    \end{equation*}

    %    Für einen Primärhashtafeleintrag mit $l_i$ Elementen wählt man $m_i > 2c \cdot \binom{l_i}{2}$.
    %    Das bedeutet, man kann $h_i$ injektiv in $\mathcal{O}(\binom{l_i}{2})$-Zeit finden.
    %
    %    \vspace*{0.3cm}
    %    Der Platzverbrauch aller Sekundärhashtafeln ist
    %    \begin{equation*}
    %        \begin{align*}
    %            \sum_{i=0}^{m-1} 2c \cdot \binom{l_i}{2} &= 2c \cdot \sum_{i=0}^{m-1}\binom{l_i}{2} \\\nonumber
    %            &= 2c \cdot \text{ Anzahl der Kollisionen von } h \\\nonumber
    %            &= 2c \cdot c_S(h) \\\nonumber
    %        \end{align*}
    %    \end{equation*}
    %    Außerdem fällt auf
    %    \begin{equation*}
    %        \Rightarrow  c_S(h) &= \sum_{i=0}^{m-1}\binom{l_i}{2} \approx n
    %    \end{equation*}
    %
    %    Für die Wahl $m=c \cdot n$ ergibt sich
    %    \begin{equation*}
    %        E(c_S(n)) \leq \binom{n}{2} \cdot \frac{c}{m} = \frac{\cancel{n} \cdot (n-1) \cdot \cancel{c}}{2 \cdot \cancel{n} \cdot \cancel{c}} = \frac{n-1}{2}
    %    \end{equation*}
    %
    %    Das heißt, der Platz aller Sekundärhashtafeln entspricht der Anzahl der Kollisionen in der Primärhashtafel.

    \textbf{\textsf{Korollar:}} Falls $m > \frac{n-1}{2} \cdot c$, dann existiert ein $h \in \mathcal{H}$ mit $c_S(h) \leq n$.

    \vspace*{0.3cm}
    \textbf{\textsf{Beweis:}} Für ein zufälliges $h \in \mathcal{H}$ gilt
    \begin{equation*}
        \begin{align*}
            E(c_S(h)) &\leq \binom{n}{2} \cdot \frac{c}{m} \\\nonumber
            & = \frac{1}{2} \cdot \frac{n \cdot (n-1)}{\frac{n-1}{2}} \\\nonumber
            & = n
        \end{align*}
    \end{equation*}\proofend

    \newpage
    \textbf{\textsf{Korollar:}} Falls $m > (n-1) \cdot 2$, dann gilt für mindestens die Hälfte aller $h \in \mathcal{H}$
    \begin{equation*}
        c_S(h) \leq n
    \end{equation*}

    \vspace*{0.3cm}
    \textbf{\textsf{Beweis:}} Wie letztes Mal.\proofend

    \vspace*{0.6cm}
    Damit ist gezeigt, dass eine Primärhashfunktion $h$ in erwarteter $\mathcal{O}(n+m)$ Zeit mit $c_S(h)\leq n$ gefunden werden kann. Die Hashtafelgröße ist dabei $m = \mathcal{O}(n)$.

    \vspace*{0.3cm}
    Seien nun alle Elemente aus $S$, die von der Primärhashfunktion $h$ in den $i$-ten Beutel gehasht werden folgendermaßen definiert:
    \begin{equation*}
        B_i(h) = h |^{-1}_S(i) = \{ x \in S\ |\ h(x)=i \}
    \end{equation*}

    Es gilt außerdem
    \begin{equation*}
        S_i(h) = \left| B_i(h) \right|
    \end{equation*}

    und
    \begin{equation*}
        c_S(h) = \sum_{i=0}^{m-1}\binom{S_i(h)}{2}
    \end{equation*}

    Da $h$ so gewählt wurde, dass $c_S(h) \leq n$, gilt auch
    \begin{equation*}
        \sum_{i=0}^{m-1}\binom{S_i(h)}{2} \leq n
    \end{equation*}

    Für jeden Hashbeutel $B_i(h)$ erzeugt man sowohl eine Sekundärhashfunktion
    als auch eine Sekundärhashtafel mit der Größe $m_i > 2c \cdot \binom{S_i(h)}{2}$, die den Inhalt von $B_i(h)$ injektiv hasht.
    Der Platzverbrauch und die Konstruktionszeit für einen Hashbeutel $B_i(h)$ ist
    \begin{equation*}
        \mathcal{O}\left( 2c \cdot \binom{S_i(h)}{2} \right) = \mathcal{O}\left( \binom{S_i(h)}{2} \right)
    \end{equation*}

    Damit kann nun die Gesamtgröße und -konstruktionszeit aller Sekundärhashtabellen und \mbox{-funktionen} eingeordnet werden
    \begin{equation*}
        \begin{align*}
            \mathcal{O}\left( \sum_{i=0}^{m-1} c \cdot \binom{S_i(h)}{2} \right) &= \mathcal{O}\left(c \cdot \underbrace{\sum_{i=0}^{m-1} \binom{S_i(h)}{2}}_{c_S(h) < n} \right) \\\nonumber
            & = \mathcal{O}(c \cdot n)
        \end{align*}
    \end{equation*}

    \newpage
    \section*{Hashing Zusammenfassung}
    Es sind $S \leq U, |S| =n$ gegeben.
    Außerdem kann man eine $c$-universelle Hashfunktion $h: U \rightarrow \{ 0,1, \cdots, m \}$ universell samplen.
    Perfektes Hashing zusammengefasst ist

    \begin{table}[H]
        \centering
        \begin{tabular}{L{2cm}lL{8cm}R{2cm}}
            \\ [-2.5ex]
            Primär-hashing & 1. & Finde $h:U \rightarrow \{ 0,1, \cdots, c \cdot n \}$, die auf $S$ maximal $n$ Kollisionen produziert. & $\mathcal{O}(c \cdot n)$-Zeit \\
            \\ [0ex]
            & 2.& Bestimme für $i= \{ 0, 1, \cdots c \cdot n \}$ das $B_i(h) = \{ x \in S\ |\ h(x)=i \}$. & \\
            \\[0ex]
            Sekundär-hashing & 3.& Für jedes $i=0, 1, \cdots, c \cdot n$ mit $|B_i(h)| > 1$ finde $h_i$ mit
            $U \rightarrow \left[ 0,1, \cdots, \binom{S_i(h)}{2} \cdot c \right]$, sodass $h_i$ injektiv ist für $B_i(h)$.& $\mathcal{O}(c \cdot n)$-Zeit\\
        \end{tabular}
    \end{table}

    \section*{Operationen auf einer gehashten Datenstruktur}
    \subsection*{Lookup eines Elements - \texttt{lookup}}
    Man hasht $x$ mit der Primärhashfunktion $h: h(x)=i$.
    \begin{itemize}
        \item Falls der errechnete Platz $\leq 1$ Elemente enthält ($B_i(h)\leq1$), ist der Zugriff fertig (Man gibt das Element zurück oder man gibt zurück, dass sich kein Element an diesem Platz befindet).
        \item Sonst wendet man die entsprechende Sekundärhashfunktion $h_i$ auf $x$ an. Unter $h_i(x)$ ist sicher ein Hashtafeleintrag mit $\leq1$ Elementen.
    \end{itemize}
    $\Rightarrow$ Die Zugriffskosten sind in $\mathcal{O}(1)$.

    \subsection*{Einfügen eines Elements - \texttt{insert}}
    Falls ein \texttt{insert} die Anzahl an Kollisionen der Primär- oder Sekundärhashfunktion zu sehr erhöht, wird alles komplett neu gehasht (Neukonstruktion).
    Das ist sehr teuer. Man kann jedoch zeigen, dass das bei universellen Hashfunktionen nicht allzu oft passiert (etwa einmal pro Verdopplung der Schlüsselmenge).

    $\Rightarrow$ Die amortisierten Kosten pro \texttt{insert} sind in $\mathcal{O}(1)$.

    \subsection*{Entfernen eines Elements - \texttt{remove}}
    Falls ein \texttt{remove} den Platzverbrauch relativ zu $|S|$ zu groß werden lässt, wird ebenfalls alles komplett neu gehasht. Das passiert etwa einmal pro Halbierung der Schlüsselmenge.

    $\Rightarrow$ Die amortisierten Kosten pro \texttt{remove} sind in $\mathcal{O}(1)$.

    \newpage
    \section*{Cuckoo-Hashing}
    Cuckoo-Hashing ist eine einfache Alternative zu Zweistufigem perfekten Hashing. Gegeben sind eine Hashtafel der Größe $m$
    sowie zwei Hashfunktionen $h_1,h_2$ aus der c-universellen Familie von Hashfunktionen $\mathcal{H}$.
    Jeder Hashtafeleintrag hat Platz für genau ein Element.

    \subsection*{Einfügen eines Elements - \texttt{insert}}
    Man hasht ein Element $x$  mit $h_1$ und überprüft, ob der Platz in der Hashtafel frei ist.
    Falls der Platz frei ist, fügt man $x$ ein und ist fertig.
    Falls ein Element $y$ den Platz schon belegt ($h_i(y) = h_1(x)$), wird $y$ rausgeworfen und $x$ nimmt seinen Platz ein.
    Das Element $y$ wird nun mittels $h_{3-i}(y)$ (also jeweils der anderen Hashfunktion) ein neuer Platz zugewiesen.
    Falls der errechnete Platz frei ist, fügt man $y$ ein und ist fertig.
    Falls ein Element $z$ den Platz schon belegt (es also gilt: $h_j(z) = h_{3-i}(y)$), wird $z$ rausgeworfen und $y$ nimmt seinen Platz ein.
    Nun wird wieder versucht, mit $h_{3-j}(z)$ einen neuen Platz für das Element $z$ zu finden $\ldots$

    $\Rightarrow$ Die amortisierten Kosten pro \texttt{insert} sind in $\mathcal{O}(1)$.

    \vspace*{0.3cm}
    \textbf{\textsf{Problem:}} Das Einfügen kann lange Tauschsequenzen beziehungsweise Zyklen zur Folge haben.
    Wenn das passiert, muss alles neu gehasht werden.
    Falls $h_1,h_2$ aus dem Beutel $c$-universeller Hashfunktionen gesampelt und die Hashtafel hinreichend groß
    gewählt wurde ($m > 2n$), ist die Wahrscheinlichkeit sehr gering, dass neu gehasht werden muss.

    \subsection*{Entfernen eines Elements - \texttt{remove}}
    Man entfernt die Elemente bei $h_1(x)$ und $h_2(x)$.
    Falls $|S|$ im Vergleich zum aktuellen $m$ zu groß wird, muss neu gehasht werden.

    $\Rightarrow$ Die amortisierten Kosten pro \texttt{remove} sind in $\mathcal{O}(1)$.

    \subsection*{Lookup eines Elements - \texttt{lookup}}
    Man schaut bei $h_1(x)$ und $h_2(x)$ nach.

    $\Rightarrow$ Die Kosten pro \texttt{lookup} sind in $\mathcal{O}(1)$.

\end{document}