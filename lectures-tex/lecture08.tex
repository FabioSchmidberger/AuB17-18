\documentclass{scrartcl}% siehe <http://www.komascript.de>
% packages ---------------------------------------------------------------------------------------------------- packages
\usepackage[margin=2.5cm]{geometry}
\usepackage[utf8]{inputenc}
\usepackage[T1]{fontenc}
\usepackage[ngerman]{babel}
\usepackage{lmodern}
\usepackage{amsmath}
\usepackage{amssymb}
\usepackage{amsfonts}
\usepackage{cancel}
\usepackage{listings}
\usepackage{multirow}
\usepackage{hhline}
\usepackage{tikz}
\usepackage{pgfplots}
\usepackage[all,dvips,arc,curve,color,frame]{xy}
\usepackage[yyyymmdd,hhmm]{datetime}
\usepackage{float}
\usepackage{hyperref}
\usepackage[table]{xcolor,colortbl}
\usepackage{graphicx}
\usepackage{subcaption}
\usepackage{tabularx}
\usepackage{footnote}
\usepackage{multicol}
\usepackage{stackengine}
\usepackage[norndcorners,customcolors,shade]{hf-tikz}
\usepackage{siunitx}
\usepackage{algorithm}
\usepackage[]{algorithmic}
% \usepackage[noend]{algorithmic}

% pgf plotsets -------------------------------------------------------------------------------------------- pgf plotsets
\pgfplotsset{ticks=none}
\pgfplotsset{width=10cm, compat=1.5.1}

% tikz libraries ---------------------------------------------------------------------------------------- tikz libraries
\usetikzlibrary{shapes}
\usetikzlibrary{positioning}
\usetikzlibrary{shapes.misc}
\usetikzlibrary{shapes.arrows,calc,quotes,babel}
\usetikzlibrary{positioning,arrows,chains,scopes,fit}
\usetikzlibrary{matrix,backgrounds}
\usetikzlibrary{trees}
\usetikzlibrary{calc}
\usetikzlibrary{tikzmark}

\sisetup{
locale = DE ,
per-mode = symbol
}

\algsetup{indent=2em}
\newlength\myindent
\setlength\myindent{2em}
\newcommand\bindent{%
    \begingroup
    \setlength{\itemindent}{\myindent}
    \addtolength{\algorithmicindent}{\myindent}
}
\newcommand\eindent{\endgroup}

% commands ---------------------------------------------------------------------------------------------------- commands
\setlength{\parindent}{0pt}
\newcommand{\includepic}[2]{\includegraphics[width=#2\textwidth,height=#2\textheight,keepaspectratio]{#1}}
\newcommand\circlearound[1]{%
\tikz[baseline]\node[draw,shape=circle,anchor=base] {#1} ;}
\newcommand{\emptyrow}[0]{\multicolumn{1}{c}{} & \multicolumn{1}{c}{} & \multicolumn{1}{c}{} & \multicolumn{1}{c}{} &
\multicolumn{1}{c}{} & \multicolumn{1}{c}{} & \multicolumn{1}{c}{} & \multicolumn{1}{c}{} &
\multicolumn{1}{c}{} & \multicolumn{1}{c|}{} & & \\}
\newcommand\underbracetext[1]{\raisebox{1ex}{\ensuremath{\underbrace{\hphantom{\text{#1}}}_{\text{\normalsize #1}}}}}
\newcommand{\headerline}[3]{
\subject{#2}
\title{#1}
\subtitle{#3}
\date{Letztes Update: \today \ - \currenttime \ Uhr}
\maketitle
}
\newcommand{\newproof}[2]{
\vspace*{0.3cm} \textbf{\textsf{Satz:}} #1

\vspace*{0.3cm} \textbf{\textsf{Beweis:}} #2 \vspace*{0.3cm}
}
\newcommand{\uparrow}[0]{\rlap{\scalebox{1.6}{$\uparrow$}}}
\newlength\dlf
\newcommand{\proofend}[0]{\hfill $\square$}
\newcommand\drawbi[7][4pt]{%
    \path(#2)--(#3)coordinate[at start](h1)coordinate[at end](h2);
    \draw[#4]($(h1)!#1!90:(h2)$)-- node [auto=left] {#5} ($(h2)!#1!-90:(h1)$);
    \draw[#6]($(h1)!#1!-90:(h2)$)-- node [auto=right] {#7} ($(h2)!#1!90:(h1)$);
    }

% colors -------------------------------------------------------------------------------------------------------- colors
\definecolor{amethyst}{rgb}{0.6, 0.4, 0.8}
\definecolor{anti-flashwhite}{rgb}{0.95, 0.95, 0.96}
\definecolor{applegreen}{rgb}{0.55, 0.71, 0.0}
\definecolor{babyblue}{rgb}{0.54, 0.81, 0.94}
\definecolor{bistre}{rgb}{0.24, 0.17, 0.12}
\definecolor{buff}{rgb}{0.94, 0.86, 0.51}
\definecolor{dodgerblue}{rgb}{0.12, 0.56, 1.0}
\definecolor{icterine}{rgb}{0.99, 0.97, 0.37}
\definecolor{tan}{rgb}{0.82, 0.71, 0.55}
\definecolor{upforestgreen}{rgb}{0.0, 0.27, 0.13}

% listing colors
\definecolor{listingBlue}{rgb}{0.13,0.13,1}
\definecolor{listingGreen}{rgb}{0,0.5,0}
\definecolor{listingRed}{rgb}{0.9,0,0}
\definecolor{listingGrey}{rgb}{0.46,0.45,0.48}

% listings ---------------------------------------------------------------------------------------------------- listings
\lstset{basicstyle=\ttfamily}
\lstset{literate=%
{Ö}{{\"O}}1
{Ä}{{\"A}}1
{Ü}{{\"U}}1
{ß}{{\ss}}1
{ü}{{\"u}}1
{ä}{{\"a}}1
{ö}{{\"o}}1
}
\lstset{language=Java,
showspaces=false,
showtabs=false,
breaklines=true,
showstringspaces=false,
breakatwhitespace=true,
commentstyle=\color{listingGreen},
keywordstyle=\color{listingBlue},
stringstyle=\color{listingRed},
basicstyle=\ttfamily,
moredelim=[il][\textcolor{listingGrey}]{$$},
moredelim=[is][\textcolor{listingGrey}]{\%\%}{\%\%}
}
\lstdefinelanguage{Kotlin}{
  keywords={package, as, typealias, this, super, val, var, fun, for, null, true, false, is, in, throw, return, break, continue, object, if, try, else, while, do, when, yield, typeof, yield, typeof, class, interface, enum, object, override, public, private, get, set, import, abstract, },
  keywordstyle=\color{listingBlue}\bfseries,
  ndkeywords={@Deprecated, Iterable, Int, Integer, Float, Double, String, Runnable, dynamic},
  ndkeywordstyle=\color{amethyst}\bfseries,
  emph={println, return@, forEach,},
  emphstyle={\color{orange}},
  identifierstyle=\color{black},
  sensitive=true,
  commentstyle=\color{listingGreen}\ttfamily,
  comment=[l]{//},
  morecomment=[s]{/*}{*/},
  stringstyle=\color{listingRed}\ttfamily,
  morestring=[b]",
  morestring=[s]{"""*}{*"""},
  numbers=left,
}

% misc ------------------------------------------------------------------------------------------------------------ misc
\graphicspath{ {../lectures-img/} }
\setlength{\parindent}{0pt}

\newcolumntype{C}[1]{>{\centering\arraybackslash}p{#1}} % zentriert mit Breitenangabe
\newcolumntype{L}[1]{>{\raggedright\arraybackslash}p{#1}} % rechtsbündig mit Breitenangabe
\newcolumntype{R}[1]{>{\raggedleft\arraybackslash}p{#1}} % rechtsbündig mit Breitenangabe


\begin{document}
    % ----------------------------------------------------------------------------
    \subject{Vorlesungsmitschrift}
    \title{Algorithmen und Berechenbarkeit}
    \subtitle{Vorlesung 08}
    \date{Letztes Update: \today \ - \currenttime \ Uhr}
    \maketitle
    % ----------------------------------------------------------------------------


    \subsection*{Weitere Sätze zum Wörterbuchproblem / Hashing}\label{subsec:erwarteteSuchzeit}
    \textbf{\textsf{Satz:}} Sei $x$ ein zufälliges (gleichverteiltes) Element aus $S$, dann ist die erwartete Suchzeit für $x$
    \begin{equation*}
        \mathcal{O}\left(1+\dfrac{1}{n} \sum_{i=0}^{m-1}\dfrac{l_i \cdot (l_i + 1)}{2}\right)
    \end{equation*}
    \textit{Besser wäre eine Schranke unabhängig von $l_i$, aber $\sum_{i}l_i = n$ sagt nichts über $\sum_{i}l_{i}^2$.}

    \vspace*{0.3cm}
    \textbf{\textsf{Beweis:}} Wenn $x$ das $j$-te Element seiner Liste ist, ist die Suchzeit $\mathcal{O}(1+j)$.
    Also ist die erwartete Suchzeit

    \begin{equation*}
        \mathcal{O} \left( \sum_{i=0}^{m-1}\sum_{j=1}^{l_i} \right) = \mathcal{O}\left( 1+ \frac{1}{n} \sum_{i=0}^{m-1} \frac{l_i - (l_i+1)}{2} \right)
    \end{equation*}
    \proofend

    \vspace*{0.6cm}
    \textbf{\textsf{Satz:}} Sei $S$ eine zufällige (gleichverteilte) Teilmenge aus $U$ der Größe $n$. Dann ist die erwartete Suchzeit für ein zufälliges $x \in S$
    \begin{equation*}
        \mathcal{O}\left(1+\beta \cdot \frac{3}{2} \cdot l^\beta\right)
    \end{equation*}

    \vspace*{0.3cm}
    \textbf{\textsf{Beweis:}} Wird nicht bewiesen.
    \vspace*{0.3cm}

    Die beiden letzten Fälle besagen im Prinzip: Falls $S \subseteq U$ zufällig gewählt wurde, ist alles gut.
    Praktisch ist das aber nicht der Fall.

    \section*{Universelles Hashing}
    Sei $\mathcal{H}$ eine Menge von Hashfunktionen von $U$ nach $\{0,1,\cdots, m-1\}$.
    Für $c > 1$ heißt $\mathcal{H}$ nun $c$-universell, falls für alle $x, y \in U$ mit $ x \neq y$ gilt:
    \begin{equation*}
        \frac{|\{ h \in \mathcal{H}:\ h(x) = h(y) \} |}{\mathcal{H}} \leq \frac{c}{m}
    \end{equation*}
    Die Gleichung sagt also prinzipiell aus, dass der Anteil der Funktionen, die zwei Inputparameter auf denselben Wert abbildet, nicht \textit{zu} groß ist.
    Es ist also ein Maß für die Qualität einer Menge von Hashfunktionen.

    \textbf{\textsf{Satz:}} Seien $a,b \in \{ 0,1,\cdots,N-1 \}$ und sei $h_{a,b} \longmapsto ((ax + b)\text{ mod } N)\text{ mod } m$.
    Dann ist die Klasse $\mathcal{H} = \{ h_{a,b}: 0 \leq a \leq N-1 \land 0 \leq b \leq N-1 \}$ $c$-universell mit
    \begin{equation*}
        c&= \frac{\left\lceil\frac{N}{m}\right\rceil}{\frac{N}{m}} \approx 1\\\nonumber
    \end{equation*}

    \vspace*{0.3cm}
    \textbf{\textsf{Beweis:}} Wird eine Übungsaufgabe.
    \vspace*{0.6cm}

    \textbf{\textsf{Satz:}} Benutzt man Hashing mit Verkettung und wählt $n \in \mathcal{H}$ zufällig gleichverteilt, wobei $\mathcal{H}$ $c$-universell ist, dann ist die erwartete Suchzeit
    \begin{equation*}
        \begin{align}
            \mathcal{O}(1+ c \cdot \beta)
        \end{align}
    \end{equation*}
    für beliebige Mengen $S \subseteq U$ und $|S| = n$.

    \vspace*{0.3cm}
    \textbf{\textsf{Beweis:}} Die Zeit für den Zugriff auf ein $x$
    ist $ < 1 + \underbrace{\overbrace{\#}^{\text{Anzahl}}(y \in S \text{ mit }h(x)=h(y))}_{\text{Wird jetzt gezeigt}}$.
    Dazu sei
    \begin{equation*}
        \delta_h(x,y) =
        \[ \begin{cases}
               1 & \text{falls } h(x) = h(y) \\
               0 & \text{sonst}
        \end{cases}
        \]
    \end{equation*}

    Dann ist

    \begin{equation*}
        \begin{align*}
            \tikzmarkin[green!30]{a}(0,-0.6)(1.4,0.65) \frac{1}{|\mathcal{H}|} \sum_{h \in \mathcal{H}}\tikzmarkend{a}
            \sum_{y \in S}\delta_h(x,y) &= \frac{1}{|\mathcal{H}|}\sum_{y \in S} \tikzmarkin[blue!30]{b}(0,-0.6)(2.1,0.65)\sum_{h \in \mathcal{H}}\delta_h(x,y)\tikzmarkend{b} \\\nonumber
        \end{align*}
    \end{equation*}

    wobei

    \begin{equation*}
        \begin{align*}
            \tikzmarkin[green!30]{a}(0,-0.5)(1.4,0.6) \frac{1}{|\mathcal{H}|} \sum_{h \in \mathcal{H}}\tikzmarkend{a}
            &\Rightarrow \text{Berechnet den Erwartungswert über alle möglichen Hashfunktionen}  \\\nonumber
            \tikzmarkin[blue!30]{b}(0,-0.5)(2.025,0.4)\sum_{h \in \mathcal{H}}\delta_h(x,y)\tikzmarkend{b}
            &\Rightarrow \text{Anzahl der Hashfunktionen, die $x$ und $y$ auf denselben Wert hashen:} \\\nonumber
            & \phantom{\Rightarrow} \rightarrow \text{Für } x = y: \qquad |\mathcal{H}|\\\nonumber
            & \phantom{\Rightarrow} \rightarrow \text{Für } x \neq y: \qquad \frac{c}{m} \cdot |\mathcal{H}|,
            \text{ da $\mathcal{H}\ c$-universell ist }\\\nonumber
        \end{align*}
    \end{equation*}

    Also

    \begin{equation*}
        \begin{align*}
            \frac{1}{|\mathcal{H}|} \sum_{h \in \mathcal{H}} \sum_{y \in S}\delta_h(x,y) &= \frac{1}{|\mathcal{H}|}\sum_{y \in S} \sum_{h \in \mathcal{H}}\delta_h(x,y) \\\nonumber
            & \leq \sum_{y \in S}\left[\text{\texttt{\textcolor{listingBlue}{if}} $(x=y)$ $1$ \texttt{\textcolor{listingBlue}{else}} } \frac{c}{m}\right] \\\nonumber
            & \leq \begin{cases}
                       1 + \frac{c \cdot (n-1)}{m} & x \in S \\
                       \frac{c}{m} \cdot n & x \notin
            \end{cases} \\\nonumber
            & \leq 1 + c \cdot \beta \\\nonumber
        \end{align*}
    \end{equation*}
    \textit{Besser wäre nicht nur eine erwartete sondern auch eine Worst-Case-Zugriffszeit von $\mathcal{O}(1)$, aber das kommt später.}
    \proofend

    \section*{Perfektes Hashing}\label{sec:perfektesHashing}
    Man möchte eine Hashfunktion $h$ mit $h(x) \neq h(y)$ für alle $x\neq y \in S$ finden.
    Außerdem soll $h$ eine Hashtafel der Größe $m=\mathcal{O}(|S|)=\mathcal{O}(n)$ bilden.
    Zusammengefasst: Man möchte eine Hashfunktion finden, die zwei unterschiedliche Inputparamter auch auf zwei unterschiedliche Werte mappt.
    Außerdem soll nur linear viel Platz verbraucht werden.

    \subsection*{Ansatz 1: Einstufiges perfektes Hashing}
    Die Anzahl der Kollisionen, die eine Hashfunktion $h$ für eine Schlüsselmenge $S$ erzeugt, ist durch die folgende Funktion beschrieben:
    \begin{equation*}
        c_S(h) = \left| \{ x,y \} \in \binom{S}{2} : h(x)=h(y)\right|
    \end{equation*}
    Das Ziel sind $0$ Kollisionen, also
    \begin{equation*}
        c_S(h)=0 \quad \Leftrightarrow \quad h \underbrace{|_s}_{\substack{\text{Einge-} \\ \text{schränkt}}} \text{ injektiv}
    \end{equation*}

    \vspace*{0.4cm}
    \textbf{\textsf{Satz:}} Für ein zufälliges $h \in \mathcal{H}$, wobei $\mathcal{H}$ $c$-universell ist, gilt:
    \begin{equation*}
        E(c_S(h)) \leq \left \binom{n}{2} \cdot \frac{c}{m}\right
    \end{equation*}

    \vspace*{0.3cm}
    \textbf{\textsf{Beweis:}} Sei
    \begin{equation*}
        c_S(h)=\sum_{\{x,y \} \in \binom{S}{2}}^{}\delta_h(x,y)
    \end{equation*}

    und weiter
    \begin{equation*}
        \begin{align*}
            E(c_S(h)) &=\sum_{\{x,y \} \in \binom{S}{2}}^{}E(\delta_h(x,y))=\sum_{\{x,y \} \in \binom{S}{2}}^{}\text{Pr}(\overbrace{\delta_h(x,y)=1}^{\leq \frac{c}{m}}) \\\nonumber
            &\leq \binom{|S|}{2} \cdot \frac{c}{m} = \binom{n}{2} \cdot \frac{c}{m}
        \end{align*}
    \end{equation*}\proofend
    \vspace*{0.6cm}

    \textbf{\textsf{Korollar:}} Sei $\mathcal{H}$ eine $c$-universelle Familie von Hashfunktionen. Für $m > c \cdot \binom{n}{2}$ existiert ein $h \in \mathcal{H}$ mit $h|_s$ injektiv.

    \vspace*{0.3cm}
    \textbf{\textsf{(Probabilistischer) Beweis:}} Durch Einsetzen erhält man
    \begin{equation*}
        E(c_S(h)) &\leq \binom{n}{2} \cdot \frac{c}{m} < 1
    \end{equation*}
    Das bedeutet, \textbf{es gibt mindestens eine Hashfunktion $h$ mit keinen Kollisionen}.
    \proofend

    \newpage
    \subsection*{Probleme mit einstufigem perfekten Hashing}
    \begin{enumerate}
        \item Der obige Satz ist nicht konstruktiv:
        Ein zufällig gewähltes $\mathcal{H}$ könnte nur mit extrem geringer Wahrscheinlichkeit injektiv sein.
        Das würde bedeuten, man bräuchte sehr viele Versuche, bis man ein injektives $h$ erwischt (\textit{Da lässt sich was machen}).
        \item Ein Platzbedarf von $n^2$ ist schlecht (\textit{Dieser Nachteil bleibt bestehen}).
    \end{enumerate}

    %    \vspace*{0.3cm}
    \textbf{\textsf{Korollar zu Problem 1:}} Falls $m > 2 \cdot c \cdot \binom{n}{2}$, kann in erwarteter $\mathcal{O}(n+m)$ Zeit eine injektive Hashfunktion gefunden werden.

    \vspace*{0.3cm}
    \textbf{\textsf{Beweis:}} Man erhält
    \begin{equation*}
        \begin{align*}
            E(c_S(h)) &\leq \binom{n}{2} \cdot \frac{c}{m} &< \cancel{\binom{n}{2}} \cdot \frac{\cancel{c}}{2 \cdot \cancel{c} \cdot \cancel{\binom{n}{2}}} \\\nonumber
            & = \frac{1}{2}
        \end{align*}
    \end{equation*}
    Mit der Markov-Ungleichung erhält man nun
    \begin{equation*}
        \begin{align*}
            \text{Pr}(X \geq c) &\leq \frac{E(X)}{c} \\\nonumber
            \text{Pr}(c_S(h) \geq 1) &\leq \frac{\frac{1}{2}}{1} = \frac{1}{2}
        \end{align*}
    \end{equation*}\proofend

    \vspace*{0.3cm}
    Das bedeutet, man erhält mit einer Wahrscheinlichkeit von $\geq \frac{1}{2}$ eine Hashfunktion $h$, die keine Kollisionen provoziert.

    \vspace*{0.3cm}
    Beim einstufigen perfekten Hashing kann ein Zugriff in $\mathcal{O}(1)$ erzielt werden, man müsste aber quadratisch viel Platz (in $|S|$) dafür aufwenden.

\end{document}