\documentclass{scrartcl}% siehe <http://www.komascript.de>
% packages ---------------------------------------------------------------------------------------------------- packages
\usepackage[margin=2.5cm]{geometry}
\usepackage[utf8]{inputenc}
\usepackage[T1]{fontenc}
\usepackage[ngerman]{babel}
\usepackage{lmodern}
\usepackage{amsmath}
\usepackage{amssymb}
\usepackage{amsfonts}
\usepackage{cancel}
\usepackage{listings}
\usepackage{multirow}
\usepackage{hhline}
\usepackage{tikz}
\usepackage{pgfplots}
\usepackage[all,dvips,arc,curve,color,frame]{xy}
\usepackage[yyyymmdd,hhmm]{datetime}
\usepackage{float}
\usepackage{hyperref}
\usepackage[table]{xcolor,colortbl}
\usepackage{graphicx}
\usepackage{subcaption}
\usepackage{tabularx}
\usepackage{footnote}
\usepackage{multicol}
\usepackage{stackengine}
\usepackage[norndcorners,customcolors,shade]{hf-tikz}
\usepackage{siunitx}
\usepackage{algorithm}
\usepackage[]{algorithmic}
% \usepackage[noend]{algorithmic}

% pgf plotsets -------------------------------------------------------------------------------------------- pgf plotsets
\pgfplotsset{ticks=none}
\pgfplotsset{width=10cm, compat=1.5.1}

% tikz libraries ---------------------------------------------------------------------------------------- tikz libraries
\usetikzlibrary{shapes}
\usetikzlibrary{positioning}
\usetikzlibrary{shapes.misc}
\usetikzlibrary{shapes.arrows,calc,quotes,babel}
\usetikzlibrary{positioning,arrows,chains,scopes,fit}
\usetikzlibrary{matrix,backgrounds}
\usetikzlibrary{trees}
\usetikzlibrary{calc}
\usetikzlibrary{tikzmark}

\sisetup{
locale = DE ,
per-mode = symbol
}

\algsetup{indent=2em}
\newlength\myindent
\setlength\myindent{2em}
\newcommand\bindent{%
    \begingroup
    \setlength{\itemindent}{\myindent}
    \addtolength{\algorithmicindent}{\myindent}
}
\newcommand\eindent{\endgroup}

% commands ---------------------------------------------------------------------------------------------------- commands
\setlength{\parindent}{0pt}
\newcommand{\includepic}[2]{\includegraphics[width=#2\textwidth,height=#2\textheight,keepaspectratio]{#1}}
\newcommand\circlearound[1]{%
\tikz[baseline]\node[draw,shape=circle,anchor=base] {#1} ;}
\newcommand{\emptyrow}[0]{\multicolumn{1}{c}{} & \multicolumn{1}{c}{} & \multicolumn{1}{c}{} & \multicolumn{1}{c}{} &
\multicolumn{1}{c}{} & \multicolumn{1}{c}{} & \multicolumn{1}{c}{} & \multicolumn{1}{c}{} &
\multicolumn{1}{c}{} & \multicolumn{1}{c|}{} & & \\}
\newcommand\underbracetext[1]{\raisebox{1ex}{\ensuremath{\underbrace{\hphantom{\text{#1}}}_{\text{\normalsize #1}}}}}
\newcommand{\headerline}[3]{
\subject{#2}
\title{#1}
\subtitle{#3}
\date{Letztes Update: \today \ - \currenttime \ Uhr}
\maketitle
}
\newcommand{\newproof}[2]{
\vspace*{0.3cm} \textbf{\textsf{Satz:}} #1

\vspace*{0.3cm} \textbf{\textsf{Beweis:}} #2 \vspace*{0.3cm}
}
\newcommand{\uparrow}[0]{\rlap{\scalebox{1.6}{$\uparrow$}}}
\newlength\dlf
\newcommand{\proofend}[0]{\hfill $\square$}
\newcommand\drawbi[7][4pt]{%
    \path(#2)--(#3)coordinate[at start](h1)coordinate[at end](h2);
    \draw[#4]($(h1)!#1!90:(h2)$)-- node [auto=left] {#5} ($(h2)!#1!-90:(h1)$);
    \draw[#6]($(h1)!#1!-90:(h2)$)-- node [auto=right] {#7} ($(h2)!#1!90:(h1)$);
    }

% colors -------------------------------------------------------------------------------------------------------- colors
\definecolor{amethyst}{rgb}{0.6, 0.4, 0.8}
\definecolor{anti-flashwhite}{rgb}{0.95, 0.95, 0.96}
\definecolor{applegreen}{rgb}{0.55, 0.71, 0.0}
\definecolor{babyblue}{rgb}{0.54, 0.81, 0.94}
\definecolor{bistre}{rgb}{0.24, 0.17, 0.12}
\definecolor{buff}{rgb}{0.94, 0.86, 0.51}
\definecolor{dodgerblue}{rgb}{0.12, 0.56, 1.0}
\definecolor{icterine}{rgb}{0.99, 0.97, 0.37}
\definecolor{tan}{rgb}{0.82, 0.71, 0.55}
\definecolor{upforestgreen}{rgb}{0.0, 0.27, 0.13}

% listing colors
\definecolor{listingBlue}{rgb}{0.13,0.13,1}
\definecolor{listingGreen}{rgb}{0,0.5,0}
\definecolor{listingRed}{rgb}{0.9,0,0}
\definecolor{listingGrey}{rgb}{0.46,0.45,0.48}

% listings ---------------------------------------------------------------------------------------------------- listings
\lstset{basicstyle=\ttfamily}
\lstset{literate=%
{Ö}{{\"O}}1
{Ä}{{\"A}}1
{Ü}{{\"U}}1
{ß}{{\ss}}1
{ü}{{\"u}}1
{ä}{{\"a}}1
{ö}{{\"o}}1
}
\lstset{language=Java,
showspaces=false,
showtabs=false,
breaklines=true,
showstringspaces=false,
breakatwhitespace=true,
commentstyle=\color{listingGreen},
keywordstyle=\color{listingBlue},
stringstyle=\color{listingRed},
basicstyle=\ttfamily,
moredelim=[il][\textcolor{listingGrey}]{$$},
moredelim=[is][\textcolor{listingGrey}]{\%\%}{\%\%}
}
\lstdefinelanguage{Kotlin}{
  keywords={package, as, typealias, this, super, val, var, fun, for, null, true, false, is, in, throw, return, break, continue, object, if, try, else, while, do, when, yield, typeof, yield, typeof, class, interface, enum, object, override, public, private, get, set, import, abstract, },
  keywordstyle=\color{listingBlue}\bfseries,
  ndkeywords={@Deprecated, Iterable, Int, Integer, Float, Double, String, Runnable, dynamic},
  ndkeywordstyle=\color{amethyst}\bfseries,
  emph={println, return@, forEach,},
  emphstyle={\color{orange}},
  identifierstyle=\color{black},
  sensitive=true,
  commentstyle=\color{listingGreen}\ttfamily,
  comment=[l]{//},
  morecomment=[s]{/*}{*/},
  stringstyle=\color{listingRed}\ttfamily,
  morestring=[b]",
  morestring=[s]{"""*}{*"""},
  numbers=left,
}

% misc ------------------------------------------------------------------------------------------------------------ misc
\graphicspath{ {../lectures-img/} }
\setlength{\parindent}{0pt}

\newcolumntype{C}[1]{>{\centering\arraybackslash}p{#1}} % zentriert mit Breitenangabe
\newcolumntype{L}[1]{>{\raggedright\arraybackslash}p{#1}} % rechtsbündig mit Breitenangabe
\newcolumntype{R}[1]{>{\raggedleft\arraybackslash}p{#1}} % rechtsbündig mit Breitenangabe


\usepackage{diagbox}

\definecolor{cadmiumred}{rgb}{0.89, 0.0, 0.13}
\definecolor{cadmiumorange}{rgb}{0.93, 0.53, 0.18}
\definecolor{cadmiumyellow}{rgb}{1.0, 0.96, 0.0}
\definecolor{cadmiumgreen}{rgb}{0.0, 0.42, 0.24}
\definecolor{blizzardblue}{rgb}{0.67, 0.9, 0.93}
\definecolor{persianblue}{rgb}{0.11, 0.22, 0.73}
\definecolor{darkviolet}{rgb}{0.58, 0.0, 0.83}

\tikzset{node cadmiumred/.style={circle,fill=cadmiumred!50,draw,minimum size=0.75cm,inner sep=0pt},}
\tikzset{node cadmiumorange/.style={circle,fill=cadmiumorange!50,draw,minimum size=0.75cm,inner sep=0pt},}
\tikzset{node cadmiumyellow/.style={circle,fill=cadmiumyellow!50,draw,minimum size=0.75cm,inner sep=0pt},}
\tikzset{node cadmiumgreen/.style={circle,fill=cadmiumgreen!50,draw,minimum size=0.75cm,inner sep=0pt},}
\tikzset{node blizzardblue/.style={circle,fill=blizzardblue!50,draw,minimum size=0.75cm,inner sep=0pt},}
\tikzset{node persianblue/.style={circle,fill=persianblue!50,draw,minimum size=0.75cm,inner sep=0pt},}
\tikzset{node darkviolet/.style={circle,fill=darkviolet!20,draw,minimum size=0.4cm,inner sep=0pt},}
\tikzset{node white/.style={circle,draw,minimum size=0.75cm,inner sep=0pt},}
\tikzset{node smwhite/.style={circle,draw,minimum size=0.4cm,inner sep=0pt},}



\begin{document}
    \headerline{Algorithmen und Berechenbarkeit}{Vorlesungsmitschrift}{Vorlesung 21}

    \textbf{\textsf{Satz:}} SAT $\leq_p$ DHC

    \vspace*{0.3cm}
    \textbf{\textsf{Beweis:}} Sei eine KNF-Formel $\phi$ gegeben.
    Man konstruiert daraus einen gerichteten Graphen $G$, der genau dann einen Hamiltonkreis hat,
    wenn $\phi$ erfüllbar ist. Seien $x_1, x_2, \dots, x_N$ die Variablen
    und $c_1, c_2, \dots, c_M$ die Klauseln von $\phi$.
    Für jede Variable $x_i$ enthält $G$ einen Subgraph $G_i$ - genannt Diamantgadget:

    \begin{figure}[H]
        \centering
        \tikzset{arrow/.style={-latex}}
        \begin{tikzpicture}[thick, scale=0.75]
            \node[node white] (n1) at (0,0) {$s_i$};
            \node[node cadmiumorange] (n2) at (-3,-3) {$l_i$};
            \node[node cadmiumgreen] (n3) at (3,-3) {$r_i$};
            \node[node white] (n4) at (0,-6) {$t_i$};

            \foreach \from/\to in {n1/n2,n1/n3,n2/n4,n3/n4}
            \draw[arrow] (\from) -> (\to);

            \node[node darkviolet] (n5) at (-1.75,-3) {};
            \node[node darkviolet] (n6) at (-0.875,-3) {};
            \node[node darkviolet] (n7) at (0,-3) {};
            \node[node darkviolet] (n8) at (0.875,-3) {};
            \node[node darkviolet] (n9) at (1.75,-3) {};

            \foreach \from/\to in {n2/n5,n5/n6,n6/n7,n7/n8,n8/n9,n9/n3}
            \drawbi{\from}{\to}{->, thin}{}{<-, thin}{};

            \draw[thick, cadmiumred, rounded corners=8pt, arrow]
            (-0.5,0) -- (-3,-2.65) -- (3, -2.65) -- (0.35, -5.65);

            \draw[thick, persianblue, rounded corners=8pt, arrow]
            (0.5,0) -- (3,-3.35) -- (-3, -3.35) -- (-0.35, -5.65);

            \node (legend) [style={draw=none},align=left] at (0, -7.5) { $\Rightarrow t_i = s_{i+1} \dots$};

        \end{tikzpicture}
    \end{figure}

    Diese Diamantgadgets werden verbunden,
    indem $t_i$ und $s_{i+1}$, $\forall i=1, \dots, n-1$ und $t_N$ und $S_1$ identifiziert werden.
    Eine Rundtour in $G$ muss die Gadgets nacheinander besuchen.
    Jedes Gadget kann von links nach rechts oder von rechts nach links durchlaufen werden.
    Im ersteren Fall interpretiert man das als $x_i =$ \texttt{true}, im letzteren als $x_i =$ \texttt{false}.

    Nun fügt man für jede Klausel einen Klauselknoten $c_j$ ein.
    Falls $c_j$ eine Variable $x_i$ enthält, wird das Gadget $G_i$ mit $c_j$ auf geeignete Art und Weise verbunden.
    Die doppelt verkettete Liste in Gadget $G_i$ enthält $2 \cdot k +1$ Knoten, falls $x_i$ in der $k$-Klausel vorkommt.

    \vspace*{0.3cm}
    Bsp: $k=2$

    \begin{figure}[H]
        \centering
        \tikzset{arrow/.style={-latex}}
        \begin{tikzpicture}[thick, scale=0.75]

            \node[node cadmiumorange] (n2) at (-6,-0) {$l_i$};
            \node[node cadmiumgreen] (n3) at (6,-0) {$r_i$};

            \node[node darkviolet] (n5) at (-4,-0) {};
            \node[node darkviolet] (n6) at (-2,-0) {};
            \node[node darkviolet] (n7) at (0,-0) {};
            \node[node darkviolet] (n8) at (2,-0) {};
            \node[node darkviolet] (n9) at (4,-0) {};

            \node[node blizzardblue] (n1) at (-4,-2) {$c_j$};

            \draw [->] (n6) to[bend left] (n1);
            \draw [->] (n1) to[bend right] (n7);

            \node [node,,align=center, text width=4cm] at (-1, 1.2){Vorkommen von $x_i$\\ in erster Klausel};
            \draw [cadmiumred] (-1,0) ellipse (1.5cm and 0.5cm);

            \node [node,,align=center, text width=4cm] at (3, -1.2){Vorkommen von $x_i$\\ in zweiter Klausel};
            \draw [cadmiumred] (3,0) ellipse (1.5cm and 0.5cm);

            \foreach \from/\to in {n2/n5,n5/n6,n6/n7,n7/n8,n8/n9,n9/n3}
            \drawbi{\from}{\to}{->, thin}{}{<-, thin}{};

        \end{tikzpicture}
    \end{figure}

    Wenn $x_i$ in der ersten Klausel nicht negiert vorkommt,
    fügt man eine Kante ein, die vom linken Knoten des Paars, das für diese Klausel steht, zum Klauselknoten zeigt.
    Analog wird eine Kante vom Klauselknoten zum rechten Knoten des Paars hinzugefügt.

    \vspace*{0.3cm}
    Wenn $x_i$ in der zweiten Klausel negiert vorkommt,
    fügt man eine Kante vom rechten Knoten zum Klauselknoten und von dort zum linken Knoten des Paars ein.

    \vspace*{0.3cm}
    \textbf{\textsf{Beobachtung:}} Eine Rundtour muss die Gadgets immer vollständig nacheinander besuchen (mit Abstecher zum Klauselknoten),
    da teilweises Besuchen eines Gadgets und Springen zu anderen Gadgets via Klauselknoten \textbf{\textsf{keine}} Rundtour ermöglicht.

    \vspace*{0.3cm}
    \textbf{\textsf{Zu zeigen:}}
    $G$ hat DHC genau dann, wenn $\phi$ erfüllbar ist.
    \begin{itemize}
        \item [$a)$] $G$ habe DHC, dann ist $x_{i} =$ \texttt{true}, wenn $G_i$ von links nach rechts durchlaufen wird und falsch andernfalls.
        Jeder Klauselknoten wird von genau einem Gadget $G_i$ aus besucht.
        \begin{itemize}
            \item Falls $G_i$ von links nach rechts traversiert wird ($\rightarrow x_i = $ \texttt{true}) und $x_i$
            taucht in Klausel nicht negiert auf, dann gilt: Die Klausel ist erfüllt.
            \item Falls $G_i$ von rechts nach links traversiert wird ($\rightarrow x_i = $ \texttt{false}) und $x_i$
            taucht in Klausel negiert auf, dann gilt: Die Klausel ist erfüllt.
        \end{itemize}
        \item [$b)$] Sei $B$ eine erfüllende Belegung für $\phi$. Dann bestimmt $B$, wie die Gadgets traversiert werden.
        \begin{itemize}
            \item $x_i = $\texttt{true} $\Rightarrow G_i$ wird von links nach rechts traversiert
            \item $x_i = $\texttt{false} $\Rightarrow G_i$ wird von rechts nach links traversiert
        \end{itemize}

        Die Klauselknoten besucht man vom Gadget des ersten erfüllenden Literals der Klausel.
    \end{itemize}\proofend

    \vspace*{0.3cm}
    \textbf{\textsf{Satz:}} TSP ist $\mathcal{N}\mathcal{P}$-hart.

    \vspace*{0.6cm}
    \textbf{\textsf{Lemma:}} HC $\leq_p$ TSP.

    \vspace*{0.3cm}
    \textbf{\textsf{Beweis:}}
    Man konstruiert aus der HC-Instanz $G(V,E)$ (wobei $G$ vollständig) eine TSP-Instanz $G'(V, E')$ wobei
    \begin{equation*}
        E' = \binom{V}{2} \text{ und }
        \[ \begin{cases}
               c(e) = 1 & \text{ falls } e \in E \\
               c(e) = 2 & \text{ sonst}
        \end{cases}
        \]
    \end{equation*}

    $G'(V, E')$ hat eine TSP-Tour mit einem Gewicht von $\leq |V|$ genau dann, wenn $G(V,E)$ HC hat.

\end{document}
